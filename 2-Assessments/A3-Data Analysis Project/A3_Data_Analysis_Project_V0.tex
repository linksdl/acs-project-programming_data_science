\documentclass[11pt]{article}

    \usepackage[breakable]{tcolorbox}
    \usepackage{parskip} % Stop auto-indenting (to mimic markdown behaviour)
    
    \usepackage{iftex}
    \ifPDFTeX
    	\usepackage[T1]{fontenc}
    	\usepackage{mathpazo}
    \else
    	\usepackage{fontspec}
    \fi

    % Basic figure setup, for now with no caption control since it's done
    % automatically by Pandoc (which extracts ![](path) syntax from Markdown).
    \usepackage{graphicx}
    % Maintain compatibility with old templates. Remove in nbconvert 6.0
    \let\Oldincludegraphics\includegraphics
    % Ensure that by default, figures have no caption (until we provide a
    % proper Figure object with a Caption API and a way to capture that
    % in the conversion process - todo).
    \usepackage{caption}
    \DeclareCaptionFormat{nocaption}{}
    \captionsetup{format=nocaption,aboveskip=0pt,belowskip=0pt}

    \usepackage[Export]{adjustbox} % Used to constrain images to a maximum size
    \adjustboxset{max size={0.9\linewidth}{0.9\paperheight}}
    \usepackage{float}
    \floatplacement{figure}{H} % forces figures to be placed at the correct location
    \usepackage{xcolor} % Allow colors to be defined
    \usepackage{enumerate} % Needed for markdown enumerations to work
    \usepackage{geometry} % Used to adjust the document margins
    \usepackage{amsmath} % Equations
    \usepackage{amssymb} % Equations
    \usepackage{textcomp} % defines textquotesingle
    % Hack from http://tex.stackexchange.com/a/47451/13684:
    \AtBeginDocument{%
        \def\PYZsq{\textquotesingle}% Upright quotes in Pygmentized code
    }
    \usepackage{upquote} % Upright quotes for verbatim code
    \usepackage{eurosym} % defines \euro
    \usepackage[mathletters]{ucs} % Extended unicode (utf-8) support
    \usepackage{fancyvrb} % verbatim replacement that allows latex
    \usepackage{grffile} % extends the file name processing of package graphics 
                         % to support a larger range
    \makeatletter % fix for grffile with XeLaTeX
    \def\Gread@@xetex#1{%
      \IfFileExists{"\Gin@base".bb}%
      {\Gread@eps{\Gin@base.bb}}%
      {\Gread@@xetex@aux#1}%
    }
    \makeatother

    % The hyperref package gives us a pdf with properly built
    % internal navigation ('pdf bookmarks' for the table of contents,
    % internal cross-reference links, web links for URLs, etc.)
    \usepackage{hyperref}
    % The default LaTeX title has an obnoxious amount of whitespace. By default,
    % titling removes some of it. It also provides customization options.
    \usepackage{titling}
    \usepackage{longtable} % longtable support required by pandoc >1.10
    \usepackage{booktabs}  % table support for pandoc > 1.12.2
    \usepackage[inline]{enumitem} % IRkernel/repr support (it uses the enumerate* environment)
    \usepackage[normalem]{ulem} % ulem is needed to support strikethroughs (\sout)
                                % normalem makes italics be italics, not underlines
    \usepackage{mathrsfs}
    

    
    % Colors for the hyperref package
    \definecolor{urlcolor}{rgb}{0,.145,.698}
    \definecolor{linkcolor}{rgb}{.71,0.21,0.01}
    \definecolor{citecolor}{rgb}{.12,.54,.11}

    % ANSI colors
    \definecolor{ansi-black}{HTML}{3E424D}
    \definecolor{ansi-black-intense}{HTML}{282C36}
    \definecolor{ansi-red}{HTML}{E75C58}
    \definecolor{ansi-red-intense}{HTML}{B22B31}
    \definecolor{ansi-green}{HTML}{00A250}
    \definecolor{ansi-green-intense}{HTML}{007427}
    \definecolor{ansi-yellow}{HTML}{DDB62B}
    \definecolor{ansi-yellow-intense}{HTML}{B27D12}
    \definecolor{ansi-blue}{HTML}{208FFB}
    \definecolor{ansi-blue-intense}{HTML}{0065CA}
    \definecolor{ansi-magenta}{HTML}{D160C4}
    \definecolor{ansi-magenta-intense}{HTML}{A03196}
    \definecolor{ansi-cyan}{HTML}{60C6C8}
    \definecolor{ansi-cyan-intense}{HTML}{258F8F}
    \definecolor{ansi-white}{HTML}{C5C1B4}
    \definecolor{ansi-white-intense}{HTML}{A1A6B2}
    \definecolor{ansi-default-inverse-fg}{HTML}{FFFFFF}
    \definecolor{ansi-default-inverse-bg}{HTML}{000000}

    % commands and environments needed by pandoc snippets
    % extracted from the output of `pandoc -s`
    \providecommand{\tightlist}{%
      \setlength{\itemsep}{0pt}\setlength{\parskip}{0pt}}
    \DefineVerbatimEnvironment{Highlighting}{Verbatim}{commandchars=\\\{\}}
    % Add ',fontsize=\small' for more characters per line
    \newenvironment{Shaded}{}{}
    \newcommand{\KeywordTok}[1]{\textcolor[rgb]{0.00,0.44,0.13}{\textbf{{#1}}}}
    \newcommand{\DataTypeTok}[1]{\textcolor[rgb]{0.56,0.13,0.00}{{#1}}}
    \newcommand{\DecValTok}[1]{\textcolor[rgb]{0.25,0.63,0.44}{{#1}}}
    \newcommand{\BaseNTok}[1]{\textcolor[rgb]{0.25,0.63,0.44}{{#1}}}
    \newcommand{\FloatTok}[1]{\textcolor[rgb]{0.25,0.63,0.44}{{#1}}}
    \newcommand{\CharTok}[1]{\textcolor[rgb]{0.25,0.44,0.63}{{#1}}}
    \newcommand{\StringTok}[1]{\textcolor[rgb]{0.25,0.44,0.63}{{#1}}}
    \newcommand{\CommentTok}[1]{\textcolor[rgb]{0.38,0.63,0.69}{\textit{{#1}}}}
    \newcommand{\OtherTok}[1]{\textcolor[rgb]{0.00,0.44,0.13}{{#1}}}
    \newcommand{\AlertTok}[1]{\textcolor[rgb]{1.00,0.00,0.00}{\textbf{{#1}}}}
    \newcommand{\FunctionTok}[1]{\textcolor[rgb]{0.02,0.16,0.49}{{#1}}}
    \newcommand{\RegionMarkerTok}[1]{{#1}}
    \newcommand{\ErrorTok}[1]{\textcolor[rgb]{1.00,0.00,0.00}{\textbf{{#1}}}}
    \newcommand{\NormalTok}[1]{{#1}}
    
    % Additional commands for more recent versions of Pandoc
    \newcommand{\ConstantTok}[1]{\textcolor[rgb]{0.53,0.00,0.00}{{#1}}}
    \newcommand{\SpecialCharTok}[1]{\textcolor[rgb]{0.25,0.44,0.63}{{#1}}}
    \newcommand{\VerbatimStringTok}[1]{\textcolor[rgb]{0.25,0.44,0.63}{{#1}}}
    \newcommand{\SpecialStringTok}[1]{\textcolor[rgb]{0.73,0.40,0.53}{{#1}}}
    \newcommand{\ImportTok}[1]{{#1}}
    \newcommand{\DocumentationTok}[1]{\textcolor[rgb]{0.73,0.13,0.13}{\textit{{#1}}}}
    \newcommand{\AnnotationTok}[1]{\textcolor[rgb]{0.38,0.63,0.69}{\textbf{\textit{{#1}}}}}
    \newcommand{\CommentVarTok}[1]{\textcolor[rgb]{0.38,0.63,0.69}{\textbf{\textit{{#1}}}}}
    \newcommand{\VariableTok}[1]{\textcolor[rgb]{0.10,0.09,0.49}{{#1}}}
    \newcommand{\ControlFlowTok}[1]{\textcolor[rgb]{0.00,0.44,0.13}{\textbf{{#1}}}}
    \newcommand{\OperatorTok}[1]{\textcolor[rgb]{0.40,0.40,0.40}{{#1}}}
    \newcommand{\BuiltInTok}[1]{{#1}}
    \newcommand{\ExtensionTok}[1]{{#1}}
    \newcommand{\PreprocessorTok}[1]{\textcolor[rgb]{0.74,0.48,0.00}{{#1}}}
    \newcommand{\AttributeTok}[1]{\textcolor[rgb]{0.49,0.56,0.16}{{#1}}}
    \newcommand{\InformationTok}[1]{\textcolor[rgb]{0.38,0.63,0.69}{\textbf{\textit{{#1}}}}}
    \newcommand{\WarningTok}[1]{\textcolor[rgb]{0.38,0.63,0.69}{\textbf{\textit{{#1}}}}}
    
    
    % Define a nice break command that doesn't care if a line doesn't already
    % exist.
    \def\br{\hspace*{\fill} \\* }
    % Math Jax compatibility definitions
    \def\gt{>}
    \def\lt{<}
    \let\Oldtex\TeX
    \let\Oldlatex\LaTeX
    \renewcommand{\TeX}{\textrm{\Oldtex}}
    \renewcommand{\LaTeX}{\textrm{\Oldlatex}}
    % Document parameters
    % Document title
    \title{A3\_Data\_Analysis\_Project\_V0}
    
    
    
    
    
% Pygments definitions
\makeatletter
\def\PY@reset{\let\PY@it=\relax \let\PY@bf=\relax%
    \let\PY@ul=\relax \let\PY@tc=\relax%
    \let\PY@bc=\relax \let\PY@ff=\relax}
\def\PY@tok#1{\csname PY@tok@#1\endcsname}
\def\PY@toks#1+{\ifx\relax#1\empty\else%
    \PY@tok{#1}\expandafter\PY@toks\fi}
\def\PY@do#1{\PY@bc{\PY@tc{\PY@ul{%
    \PY@it{\PY@bf{\PY@ff{#1}}}}}}}
\def\PY#1#2{\PY@reset\PY@toks#1+\relax+\PY@do{#2}}

\expandafter\def\csname PY@tok@w\endcsname{\def\PY@tc##1{\textcolor[rgb]{0.73,0.73,0.73}{##1}}}
\expandafter\def\csname PY@tok@c\endcsname{\let\PY@it=\textit\def\PY@tc##1{\textcolor[rgb]{0.25,0.50,0.50}{##1}}}
\expandafter\def\csname PY@tok@cp\endcsname{\def\PY@tc##1{\textcolor[rgb]{0.74,0.48,0.00}{##1}}}
\expandafter\def\csname PY@tok@k\endcsname{\let\PY@bf=\textbf\def\PY@tc##1{\textcolor[rgb]{0.00,0.50,0.00}{##1}}}
\expandafter\def\csname PY@tok@kp\endcsname{\def\PY@tc##1{\textcolor[rgb]{0.00,0.50,0.00}{##1}}}
\expandafter\def\csname PY@tok@kt\endcsname{\def\PY@tc##1{\textcolor[rgb]{0.69,0.00,0.25}{##1}}}
\expandafter\def\csname PY@tok@o\endcsname{\def\PY@tc##1{\textcolor[rgb]{0.40,0.40,0.40}{##1}}}
\expandafter\def\csname PY@tok@ow\endcsname{\let\PY@bf=\textbf\def\PY@tc##1{\textcolor[rgb]{0.67,0.13,1.00}{##1}}}
\expandafter\def\csname PY@tok@nb\endcsname{\def\PY@tc##1{\textcolor[rgb]{0.00,0.50,0.00}{##1}}}
\expandafter\def\csname PY@tok@nf\endcsname{\def\PY@tc##1{\textcolor[rgb]{0.00,0.00,1.00}{##1}}}
\expandafter\def\csname PY@tok@nc\endcsname{\let\PY@bf=\textbf\def\PY@tc##1{\textcolor[rgb]{0.00,0.00,1.00}{##1}}}
\expandafter\def\csname PY@tok@nn\endcsname{\let\PY@bf=\textbf\def\PY@tc##1{\textcolor[rgb]{0.00,0.00,1.00}{##1}}}
\expandafter\def\csname PY@tok@ne\endcsname{\let\PY@bf=\textbf\def\PY@tc##1{\textcolor[rgb]{0.82,0.25,0.23}{##1}}}
\expandafter\def\csname PY@tok@nv\endcsname{\def\PY@tc##1{\textcolor[rgb]{0.10,0.09,0.49}{##1}}}
\expandafter\def\csname PY@tok@no\endcsname{\def\PY@tc##1{\textcolor[rgb]{0.53,0.00,0.00}{##1}}}
\expandafter\def\csname PY@tok@nl\endcsname{\def\PY@tc##1{\textcolor[rgb]{0.63,0.63,0.00}{##1}}}
\expandafter\def\csname PY@tok@ni\endcsname{\let\PY@bf=\textbf\def\PY@tc##1{\textcolor[rgb]{0.60,0.60,0.60}{##1}}}
\expandafter\def\csname PY@tok@na\endcsname{\def\PY@tc##1{\textcolor[rgb]{0.49,0.56,0.16}{##1}}}
\expandafter\def\csname PY@tok@nt\endcsname{\let\PY@bf=\textbf\def\PY@tc##1{\textcolor[rgb]{0.00,0.50,0.00}{##1}}}
\expandafter\def\csname PY@tok@nd\endcsname{\def\PY@tc##1{\textcolor[rgb]{0.67,0.13,1.00}{##1}}}
\expandafter\def\csname PY@tok@s\endcsname{\def\PY@tc##1{\textcolor[rgb]{0.73,0.13,0.13}{##1}}}
\expandafter\def\csname PY@tok@sd\endcsname{\let\PY@it=\textit\def\PY@tc##1{\textcolor[rgb]{0.73,0.13,0.13}{##1}}}
\expandafter\def\csname PY@tok@si\endcsname{\let\PY@bf=\textbf\def\PY@tc##1{\textcolor[rgb]{0.73,0.40,0.53}{##1}}}
\expandafter\def\csname PY@tok@se\endcsname{\let\PY@bf=\textbf\def\PY@tc##1{\textcolor[rgb]{0.73,0.40,0.13}{##1}}}
\expandafter\def\csname PY@tok@sr\endcsname{\def\PY@tc##1{\textcolor[rgb]{0.73,0.40,0.53}{##1}}}
\expandafter\def\csname PY@tok@ss\endcsname{\def\PY@tc##1{\textcolor[rgb]{0.10,0.09,0.49}{##1}}}
\expandafter\def\csname PY@tok@sx\endcsname{\def\PY@tc##1{\textcolor[rgb]{0.00,0.50,0.00}{##1}}}
\expandafter\def\csname PY@tok@m\endcsname{\def\PY@tc##1{\textcolor[rgb]{0.40,0.40,0.40}{##1}}}
\expandafter\def\csname PY@tok@gh\endcsname{\let\PY@bf=\textbf\def\PY@tc##1{\textcolor[rgb]{0.00,0.00,0.50}{##1}}}
\expandafter\def\csname PY@tok@gu\endcsname{\let\PY@bf=\textbf\def\PY@tc##1{\textcolor[rgb]{0.50,0.00,0.50}{##1}}}
\expandafter\def\csname PY@tok@gd\endcsname{\def\PY@tc##1{\textcolor[rgb]{0.63,0.00,0.00}{##1}}}
\expandafter\def\csname PY@tok@gi\endcsname{\def\PY@tc##1{\textcolor[rgb]{0.00,0.63,0.00}{##1}}}
\expandafter\def\csname PY@tok@gr\endcsname{\def\PY@tc##1{\textcolor[rgb]{1.00,0.00,0.00}{##1}}}
\expandafter\def\csname PY@tok@ge\endcsname{\let\PY@it=\textit}
\expandafter\def\csname PY@tok@gs\endcsname{\let\PY@bf=\textbf}
\expandafter\def\csname PY@tok@gp\endcsname{\let\PY@bf=\textbf\def\PY@tc##1{\textcolor[rgb]{0.00,0.00,0.50}{##1}}}
\expandafter\def\csname PY@tok@go\endcsname{\def\PY@tc##1{\textcolor[rgb]{0.53,0.53,0.53}{##1}}}
\expandafter\def\csname PY@tok@gt\endcsname{\def\PY@tc##1{\textcolor[rgb]{0.00,0.27,0.87}{##1}}}
\expandafter\def\csname PY@tok@err\endcsname{\def\PY@bc##1{\setlength{\fboxsep}{0pt}\fcolorbox[rgb]{1.00,0.00,0.00}{1,1,1}{\strut ##1}}}
\expandafter\def\csname PY@tok@kc\endcsname{\let\PY@bf=\textbf\def\PY@tc##1{\textcolor[rgb]{0.00,0.50,0.00}{##1}}}
\expandafter\def\csname PY@tok@kd\endcsname{\let\PY@bf=\textbf\def\PY@tc##1{\textcolor[rgb]{0.00,0.50,0.00}{##1}}}
\expandafter\def\csname PY@tok@kn\endcsname{\let\PY@bf=\textbf\def\PY@tc##1{\textcolor[rgb]{0.00,0.50,0.00}{##1}}}
\expandafter\def\csname PY@tok@kr\endcsname{\let\PY@bf=\textbf\def\PY@tc##1{\textcolor[rgb]{0.00,0.50,0.00}{##1}}}
\expandafter\def\csname PY@tok@bp\endcsname{\def\PY@tc##1{\textcolor[rgb]{0.00,0.50,0.00}{##1}}}
\expandafter\def\csname PY@tok@fm\endcsname{\def\PY@tc##1{\textcolor[rgb]{0.00,0.00,1.00}{##1}}}
\expandafter\def\csname PY@tok@vc\endcsname{\def\PY@tc##1{\textcolor[rgb]{0.10,0.09,0.49}{##1}}}
\expandafter\def\csname PY@tok@vg\endcsname{\def\PY@tc##1{\textcolor[rgb]{0.10,0.09,0.49}{##1}}}
\expandafter\def\csname PY@tok@vi\endcsname{\def\PY@tc##1{\textcolor[rgb]{0.10,0.09,0.49}{##1}}}
\expandafter\def\csname PY@tok@vm\endcsname{\def\PY@tc##1{\textcolor[rgb]{0.10,0.09,0.49}{##1}}}
\expandafter\def\csname PY@tok@sa\endcsname{\def\PY@tc##1{\textcolor[rgb]{0.73,0.13,0.13}{##1}}}
\expandafter\def\csname PY@tok@sb\endcsname{\def\PY@tc##1{\textcolor[rgb]{0.73,0.13,0.13}{##1}}}
\expandafter\def\csname PY@tok@sc\endcsname{\def\PY@tc##1{\textcolor[rgb]{0.73,0.13,0.13}{##1}}}
\expandafter\def\csname PY@tok@dl\endcsname{\def\PY@tc##1{\textcolor[rgb]{0.73,0.13,0.13}{##1}}}
\expandafter\def\csname PY@tok@s2\endcsname{\def\PY@tc##1{\textcolor[rgb]{0.73,0.13,0.13}{##1}}}
\expandafter\def\csname PY@tok@sh\endcsname{\def\PY@tc##1{\textcolor[rgb]{0.73,0.13,0.13}{##1}}}
\expandafter\def\csname PY@tok@s1\endcsname{\def\PY@tc##1{\textcolor[rgb]{0.73,0.13,0.13}{##1}}}
\expandafter\def\csname PY@tok@mb\endcsname{\def\PY@tc##1{\textcolor[rgb]{0.40,0.40,0.40}{##1}}}
\expandafter\def\csname PY@tok@mf\endcsname{\def\PY@tc##1{\textcolor[rgb]{0.40,0.40,0.40}{##1}}}
\expandafter\def\csname PY@tok@mh\endcsname{\def\PY@tc##1{\textcolor[rgb]{0.40,0.40,0.40}{##1}}}
\expandafter\def\csname PY@tok@mi\endcsname{\def\PY@tc##1{\textcolor[rgb]{0.40,0.40,0.40}{##1}}}
\expandafter\def\csname PY@tok@il\endcsname{\def\PY@tc##1{\textcolor[rgb]{0.40,0.40,0.40}{##1}}}
\expandafter\def\csname PY@tok@mo\endcsname{\def\PY@tc##1{\textcolor[rgb]{0.40,0.40,0.40}{##1}}}
\expandafter\def\csname PY@tok@ch\endcsname{\let\PY@it=\textit\def\PY@tc##1{\textcolor[rgb]{0.25,0.50,0.50}{##1}}}
\expandafter\def\csname PY@tok@cm\endcsname{\let\PY@it=\textit\def\PY@tc##1{\textcolor[rgb]{0.25,0.50,0.50}{##1}}}
\expandafter\def\csname PY@tok@cpf\endcsname{\let\PY@it=\textit\def\PY@tc##1{\textcolor[rgb]{0.25,0.50,0.50}{##1}}}
\expandafter\def\csname PY@tok@c1\endcsname{\let\PY@it=\textit\def\PY@tc##1{\textcolor[rgb]{0.25,0.50,0.50}{##1}}}
\expandafter\def\csname PY@tok@cs\endcsname{\let\PY@it=\textit\def\PY@tc##1{\textcolor[rgb]{0.25,0.50,0.50}{##1}}}

\def\PYZbs{\char`\\}
\def\PYZus{\char`\_}
\def\PYZob{\char`\{}
\def\PYZcb{\char`\}}
\def\PYZca{\char`\^}
\def\PYZam{\char`\&}
\def\PYZlt{\char`\<}
\def\PYZgt{\char`\>}
\def\PYZsh{\char`\#}
\def\PYZpc{\char`\%}
\def\PYZdl{\char`\$}
\def\PYZhy{\char`\-}
\def\PYZsq{\char`\'}
\def\PYZdq{\char`\"}
\def\PYZti{\char`\~}
% for compatibility with earlier versions
\def\PYZat{@}
\def\PYZlb{[}
\def\PYZrb{]}
\makeatother


    % For linebreaks inside Verbatim environment from package fancyvrb. 
    \makeatletter
        \newbox\Wrappedcontinuationbox 
        \newbox\Wrappedvisiblespacebox 
        \newcommand*\Wrappedvisiblespace {\textcolor{red}{\textvisiblespace}} 
        \newcommand*\Wrappedcontinuationsymbol {\textcolor{red}{\llap{\tiny$\m@th\hookrightarrow$}}} 
        \newcommand*\Wrappedcontinuationindent {3ex } 
        \newcommand*\Wrappedafterbreak {\kern\Wrappedcontinuationindent\copy\Wrappedcontinuationbox} 
        % Take advantage of the already applied Pygments mark-up to insert 
        % potential linebreaks for TeX processing. 
        %        {, <, #, %, $, ' and ": go to next line. 
        %        _, }, ^, &, >, - and ~: stay at end of broken line. 
        % Use of \textquotesingle for straight quote. 
        \newcommand*\Wrappedbreaksatspecials {% 
            \def\PYGZus{\discretionary{\char`\_}{\Wrappedafterbreak}{\char`\_}}% 
            \def\PYGZob{\discretionary{}{\Wrappedafterbreak\char`\{}{\char`\{}}% 
            \def\PYGZcb{\discretionary{\char`\}}{\Wrappedafterbreak}{\char`\}}}% 
            \def\PYGZca{\discretionary{\char`\^}{\Wrappedafterbreak}{\char`\^}}% 
            \def\PYGZam{\discretionary{\char`\&}{\Wrappedafterbreak}{\char`\&}}% 
            \def\PYGZlt{\discretionary{}{\Wrappedafterbreak\char`\<}{\char`\<}}% 
            \def\PYGZgt{\discretionary{\char`\>}{\Wrappedafterbreak}{\char`\>}}% 
            \def\PYGZsh{\discretionary{}{\Wrappedafterbreak\char`\#}{\char`\#}}% 
            \def\PYGZpc{\discretionary{}{\Wrappedafterbreak\char`\%}{\char`\%}}% 
            \def\PYGZdl{\discretionary{}{\Wrappedafterbreak\char`\$}{\char`\$}}% 
            \def\PYGZhy{\discretionary{\char`\-}{\Wrappedafterbreak}{\char`\-}}% 
            \def\PYGZsq{\discretionary{}{\Wrappedafterbreak\textquotesingle}{\textquotesingle}}% 
            \def\PYGZdq{\discretionary{}{\Wrappedafterbreak\char`\"}{\char`\"}}% 
            \def\PYGZti{\discretionary{\char`\~}{\Wrappedafterbreak}{\char`\~}}% 
        } 
        % Some characters . , ; ? ! / are not pygmentized. 
        % This macro makes them "active" and they will insert potential linebreaks 
        \newcommand*\Wrappedbreaksatpunct {% 
            \lccode`\~`\.\lowercase{\def~}{\discretionary{\hbox{\char`\.}}{\Wrappedafterbreak}{\hbox{\char`\.}}}% 
            \lccode`\~`\,\lowercase{\def~}{\discretionary{\hbox{\char`\,}}{\Wrappedafterbreak}{\hbox{\char`\,}}}% 
            \lccode`\~`\;\lowercase{\def~}{\discretionary{\hbox{\char`\;}}{\Wrappedafterbreak}{\hbox{\char`\;}}}% 
            \lccode`\~`\:\lowercase{\def~}{\discretionary{\hbox{\char`\:}}{\Wrappedafterbreak}{\hbox{\char`\:}}}% 
            \lccode`\~`\?\lowercase{\def~}{\discretionary{\hbox{\char`\?}}{\Wrappedafterbreak}{\hbox{\char`\?}}}% 
            \lccode`\~`\!\lowercase{\def~}{\discretionary{\hbox{\char`\!}}{\Wrappedafterbreak}{\hbox{\char`\!}}}% 
            \lccode`\~`\/\lowercase{\def~}{\discretionary{\hbox{\char`\/}}{\Wrappedafterbreak}{\hbox{\char`\/}}}% 
            \catcode`\.\active
            \catcode`\,\active 
            \catcode`\;\active
            \catcode`\:\active
            \catcode`\?\active
            \catcode`\!\active
            \catcode`\/\active 
            \lccode`\~`\~ 	
        }
    \makeatother

    \let\OriginalVerbatim=\Verbatim
    \makeatletter
    \renewcommand{\Verbatim}[1][1]{%
        %\parskip\z@skip
        \sbox\Wrappedcontinuationbox {\Wrappedcontinuationsymbol}%
        \sbox\Wrappedvisiblespacebox {\FV@SetupFont\Wrappedvisiblespace}%
        \def\FancyVerbFormatLine ##1{\hsize\linewidth
            \vtop{\raggedright\hyphenpenalty\z@\exhyphenpenalty\z@
                \doublehyphendemerits\z@\finalhyphendemerits\z@
                \strut ##1\strut}%
        }%
        % If the linebreak is at a space, the latter will be displayed as visible
        % space at end of first line, and a continuation symbol starts next line.
        % Stretch/shrink are however usually zero for typewriter font.
        \def\FV@Space {%
            \nobreak\hskip\z@ plus\fontdimen3\font minus\fontdimen4\font
            \discretionary{\copy\Wrappedvisiblespacebox}{\Wrappedafterbreak}
            {\kern\fontdimen2\font}%
        }%
        
        % Allow breaks at special characters using \PYG... macros.
        \Wrappedbreaksatspecials
        % Breaks at punctuation characters . , ; ? ! and / need catcode=\active 	
        \OriginalVerbatim[#1,codes*=\Wrappedbreaksatpunct]%
    }
    \makeatother

    % Exact colors from NB
    \definecolor{incolor}{HTML}{303F9F}
    \definecolor{outcolor}{HTML}{D84315}
    \definecolor{cellborder}{HTML}{CFCFCF}
    \definecolor{cellbackground}{HTML}{F7F7F7}
    
    % prompt
    \makeatletter
    \newcommand{\boxspacing}{\kern\kvtcb@left@rule\kern\kvtcb@boxsep}
    \makeatother
    \newcommand{\prompt}[4]{
        \ttfamily\llap{{\color{#2}[#3]:\hspace{3pt}#4}}\vspace{-\baselineskip}
    }
    

    
    % Prevent overflowing lines due to hard-to-break entities
    \sloppy 
    % Setup hyperref package
    \hypersetup{
      breaklinks=true,  % so long urls are correctly broken across lines
      colorlinks=true,
      urlcolor=urlcolor,
      linkcolor=linkcolor,
      citecolor=citecolor,
      }
    % Slightly bigger margins than the latex defaults
    
    \geometry{verbose,tmargin=1in,bmargin=1in,lmargin=1in,rmargin=1in}
    
    

\begin{document}
    
    \maketitle
    
    

    
    \hypertarget{p4ds-assignment-3-autumn-2020}{%
\subsection{P4DS: Assignment 3 (Autumn
2020)}\label{p4ds-assignment-3-autumn-2020}}

\hypertarget{data-analysis-project}{%
\section{Data Analysis Project}\label{data-analysis-project}}

\hypertarget{notebook-template-design-brandon-bennett-2020113}{%
\paragraph{Notebook template design: Brandon Bennett
(2020/11/3)}\label{notebook-template-design-brandon-bennett-2020113}}

\hypertarget{the-bread-basket}{%
\section{The Bread Basket}\label{the-bread-basket}}

\hypertarget{project-participants}{%
\subsubsection{Project participants:}\label{project-participants}}

\begin{itemize}
\tightlist
\item
  A. Daolin Sheng (ml192ds@leeds.ac.uk)
\item
  B. Xiangyu Li (ml19xl4@leeds.ac.uk)
\end{itemize}

    \hypertarget{project-plan}{%
\section{Project Plan}\label{project-plan}}

\hypertarget{the-data-10-marks}{%
\subsection{The Data (10 marks)}\label{the-data-10-marks}}

This project just contains a dataset and it comes from a Kaggle
web-page(\href{https://www.kaggle.com/mittalvasu95/the-bread-basket}{The
Bread Basket}), and it belongs to ``The Bread Basket'' a bakery located
in Edinburgh. This a sample dataset about online shopping. The dataset
includes more than 9000 transactions of customers who order different
items from this bakery online from 30-10-2016 to 09-04-2017 and has
20507 entries.

The dataset has 5 columns and they are `Transaction', `Item',
'date\_time', `period\_day', `weekday\_weekend' respectively. The first
column `Transaction' is the transaction id which is unique for each
order and ranges from 1 to 9684, and there is not missing column value.
The second column name is called `Item' which includes more than 20
different food goods name, such as Bread, Coffee and Jam etc, and there
is not missing column value. The third column is the date and time of
every transaction, and the format of column value is `dd-mm-yyyy hh:mm'.
the fourth column value is a period time of one day when a customer
places an order and it has two main value `morning' and `afternoon'. the
last column is called `weekday\_weekend' which only has two different
values and they represent whether these orders are on weekday or
weekend.

In conclusion, this is a simple business dataset, and There are no
missing values in this data set.

\hypertarget{project-aim-and-objectives-5-marks}{%
\subsection{Project Aim and Objectives (5
marks)}\label{project-aim-and-objectives-5-marks}}

This dataset is about the bakery online sales data, so we want to
develop a simple system which will contain three main modules. These
modules are statistics, search and recommendation respectively and they
should help the bakery manager better to operate and manage this bakery
and to increase sales. In order to implement this simple system, we can
start from the following aims.

The first aim, we want to this system can have a statistic module based
on the history data. To specific, we can get we want to get statistic
data. For example, the sales volume over a period of time, the
percentage of sales of each product or the max sales volume of a day
etc, and these statistics data can be easily visualized.

The second aim, we hope this system can have a search module which can
return different results based on the different search criteria. By
inputting a parameter `n', for instance, we can find the top `n'
products in sales volume.

The last aim is about a recommendation module. This module needs to have
a function which can recommend products as accurately as possible to
customers based on his/her purchase data. This function will be
implemented based on machine learning algorithms.

To sum up, these aims or objects will fulfill the demands of different
aspects of a system. The specific objectives following:

\hypertarget{specific-objectives}{%
\subsubsection{Specific Objective(s)}\label{specific-objectives}}

\begin{itemize}
\tightlist
\item
  \textbf{Objective 1:} Count historical sales volume(daily, monthly).
\item
  \textbf{Objective 2:} Find the top `n' popular products in a period of
  time.
\item
  \textbf{Objective 3:} Recommend product to customers
\end{itemize}

\hypertarget{system-design-5-marks}{%
\subsection{System Design (5 marks)}\label{system-design-5-marks}}

\hypertarget{architecture}{%
\subsubsection{Architecture}\label{architecture}}

\begin{figure}
\centering
\includegraphics{attachment:Project\%20Architecture-V1.png}
\caption{Project\%20Architecture-V1.png}
\end{figure}

This is a simple architecture of this project and it contains 5 steps to
complete this project. Of course, this architecture is a general data
science pipeline. The first three steps are data wrangling, data
cleansing and data preparation, and they all belong to data engineering.
The data wrangling includes identification, collection and combination
etc. These are important parts of the data wrangling to prepare for the
data cleaning. The next step is data cleansing which can handle the
common errors in a dataset and make sure the data set is valid. The
third step is data preparation, and there are special tasks, such as
one-hot encode, normalization, to prepare for a machine learning
algorithm.

The fourth step mainly uses some algorithms to process data. These
algorithms would include search algorithm, count algorithm and
recommendation algorithm. The last step can get the result according to
the function implemented in the previous step. In this project, the
output result would be an items list, or a visual diagram, or a text
report.

\hypertarget{processing-modules-and-algorithms}{%
\subsubsection{Processing Modules and
Algorithms}\label{processing-modules-and-algorithms}}

\begin{itemize}
\tightlist
\item
  Convert the column date\_time to other five different columns(the
  date\_time column will be split into other different five columns)
\item
  Implement an algorithm to find the top n popular items over period of
  time(this algorithm contains two parameters)
\item
  Convert the dataset into a feature vectors to model learning
\end{itemize}

    \hypertarget{program-code-15-marks}{%
\section{Program Code (15 marks)}\label{program-code-15-marks}}

    Import the some modules needed for this project. Data handling package
numpy and pandas and the data visualization module matplotlib and
seaborn.

    \begin{tcolorbox}[breakable, size=fbox, boxrule=1pt, pad at break*=1mm,colback=cellbackground, colframe=cellborder]
\prompt{In}{incolor}{8}{\boxspacing}
\begin{Verbatim}[commandchars=\\\{\}]
\PY{c+c1}{\PYZsh{} Import the module numpy, pandas, matplotlib, seaborn to handle and visualize data.}
\PY{k+kn}{import} \PY{n+nn}{numpy} \PY{k}{as} \PY{n+nn}{np}
\PY{k+kn}{import} \PY{n+nn}{pandas} \PY{k}{as} \PY{n+nn}{pd}
\PY{k+kn}{import} \PY{n+nn}{matplotlib}\PY{n+nn}{.}\PY{n+nn}{pyplot} \PY{k}{as} \PY{n+nn}{plt}
\PY{k+kn}{import} \PY{n+nn}{seaborn} \PY{k}{as} \PY{n+nn}{sns}
\PY{k+kn}{import} \PY{n+nn}{warnings}
\PY{n}{warnings}\PY{o}{.}\PY{n}{filterwarnings}\PY{p}{(}\PY{l+s+s2}{\PYZdq{}}\PY{l+s+s2}{ignore}\PY{l+s+s2}{\PYZdq{}}\PY{p}{)}
\end{Verbatim}
\end{tcolorbox}

    Using the pandas \texttt{read\_csv()} load the dataset and convert the
raw dataset into a DataFrame. By using pands built-in function
\texttt{head()}, \texttt{tail()}, \texttt{describe()} and \texttt{info}
get the dataser profile and then do the data cleansing.

    \begin{tcolorbox}[breakable, size=fbox, boxrule=1pt, pad at break*=1mm,colback=cellbackground, colframe=cellborder]
\prompt{In}{incolor}{9}{\boxspacing}
\begin{Verbatim}[commandchars=\\\{\}]
\PY{c+c1}{\PYZsh{} load the dataset into pandas DataFrame.}
\PY{n}{df} \PY{o}{=} \PY{n}{pd}\PY{o}{.}\PY{n}{read\PYZus{}csv}\PY{p}{(}\PY{l+s+s1}{\PYZsq{}}\PY{l+s+s1}{./bread\PYZus{}basket\PYZus{}1.csv}\PY{l+s+s1}{\PYZsq{}}\PY{p}{)}
\PY{c+c1}{\PYZsh{} load the fist 10 entries.}
\PY{n}{display}\PY{p}{(}\PY{n}{df}\PY{o}{.}\PY{n}{head}\PY{p}{(}\PY{l+m+mi}{10}\PY{p}{)}\PY{p}{)}
\PY{c+c1}{\PYZsh{} load the last 10 entries.}
\PY{n}{display}\PY{p}{(}\PY{n}{df}\PY{o}{.}\PY{n}{tail}\PY{p}{(}\PY{l+m+mi}{10}\PY{p}{)}\PY{p}{)}
\PY{c+c1}{\PYZsh{} get the info of the DataFrame.}
\PY{n}{display}\PY{p}{(}\PY{n}{df}\PY{o}{.}\PY{n}{info}\PY{p}{(}\PY{p}{)}\PY{p}{)}
\end{Verbatim}
\end{tcolorbox}

    
    \begin{verbatim}
   Transaction           Item         date_time period_day weekday_weekend
0            1          Bread  30-10-2016 09:58    morning         weekend
1            2   Scandinavian  30-10-2016 10:05    morning         weekend
2            2   Scandinavian  30-10-2016 10:05    morning         weekend
3            3  Hot chocolate  30-10-2016 10:07    morning         weekend
4            3            Jam  30-10-2016 10:07    morning         weekend
5            3        Cookies  30-10-2016 10:07    morning         weekend
6            4         Muffin  30-10-2016 10:08    morning         weekend
7            5         Coffee  30-10-2016 10:13    morning         weekend
8            5         Pastry  30-10-2016 10:13    morning         weekend
9            5          Bread  30-10-2016 10:13    morning         weekend
    \end{verbatim}

    
    
    \begin{verbatim}
       Transaction              Item         date_time period_day  \
20497         9681               Tea  09-04-2017 14:30  afternoon   
20498         9681    Spanish Brunch  09-04-2017 14:30  afternoon   
20499         9681  Christmas common  09-04-2017 14:30  afternoon   
20500         9682            Muffin  09-04-2017 14:32  afternoon   
20501         9682      Tacos/Fajita  09-04-2017 14:32  afternoon   
20502         9682            Coffee  09-04-2017 14:32  afternoon   
20503         9682               Tea  09-04-2017 14:32  afternoon   
20504         9683            Coffee  09-04-2017 14:57  afternoon   
20505         9683            Pastry  09-04-2017 14:57  afternoon   
20506         9684         Smoothies  09-04-2017 15:04  afternoon   

      weekday_weekend  
20497         weekend  
20498         weekend  
20499         weekend  
20500         weekend  
20501         weekend  
20502         weekend  
20503         weekend  
20504         weekend  
20505         weekend  
20506         weekend  
    \end{verbatim}

    
    \begin{Verbatim}[commandchars=\\\{\}]
<class 'pandas.core.frame.DataFrame'>
RangeIndex: 20507 entries, 0 to 20506
Data columns (total 5 columns):
 \#   Column           Non-Null Count  Dtype
---  ------           --------------  -----
 0   Transaction      20507 non-null  int64
 1   Item             20507 non-null  object
 2   date\_time        20507 non-null  object
 3   period\_day       20507 non-null  object
 4   weekday\_weekend  20507 non-null  object
dtypes: int64(1), object(4)
memory usage: 801.2+ KB
    \end{Verbatim}

    
    \begin{verbatim}
None
    \end{verbatim}

    
    \textbf{By parsing the dataset, because there are not errors in the
dataset and there is one dataset in this project, it is not necessary to
merge dataset and data cleansing. we diectly entry into the step three
`Data preparation'.}

the following funtion \texttt{dataset\_datetime\_split()} implement the
function which split the column into other five different columns.

    \begin{tcolorbox}[breakable, size=fbox, boxrule=1pt, pad at break*=1mm,colback=cellbackground, colframe=cellborder]
\prompt{In}{incolor}{10}{\boxspacing}
\begin{Verbatim}[commandchars=\\\{\}]
\PY{c+c1}{\PYZsh{} The input argument is a pandas DataFrame.}
\PY{k}{def} \PY{n+nf}{dataset\PYZus{}datetime\PYZus{}split}\PY{p}{(}\PY{n}{df}\PY{p}{)}\PY{p}{:}
    \PY{c+c1}{\PYZsh{} Formatting the date\PYZus{}time into the right format.}
    \PY{n}{df}\PY{p}{[}\PY{l+s+s1}{\PYZsq{}}\PY{l+s+s1}{date\PYZus{}time}\PY{l+s+s1}{\PYZsq{}}\PY{p}{]} \PY{o}{=} \PY{n}{pd}\PY{o}{.}\PY{n}{to\PYZus{}datetime}\PY{p}{(}\PY{n}{df}\PY{p}{[}\PY{l+s+s1}{\PYZsq{}}\PY{l+s+s1}{date\PYZus{}time}\PY{l+s+s1}{\PYZsq{}}\PY{p}{]}\PY{p}{,} \PY{n+nb}{format}\PY{o}{=}\PY{l+s+s2}{\PYZdq{}}\PY{l+s+si}{\PYZpc{}d}\PY{l+s+s2}{\PYZhy{}}\PY{l+s+s2}{\PYZpc{}}\PY{l+s+s2}{m\PYZhy{}}\PY{l+s+s2}{\PYZpc{}}\PY{l+s+s2}{Y }\PY{l+s+s2}{\PYZpc{}}\PY{l+s+s2}{H:}\PY{l+s+s2}{\PYZpc{}}\PY{l+s+s2}{M}\PY{l+s+s2}{\PYZdq{}}\PY{p}{)}
    \PY{c+c1}{\PYZsh{} Getting the date}
    \PY{n}{df}\PY{p}{[}\PY{l+s+s1}{\PYZsq{}}\PY{l+s+s1}{date}\PY{l+s+s1}{\PYZsq{}}\PY{p}{]}  \PY{o}{=} \PY{n}{df}\PY{p}{[}\PY{l+s+s1}{\PYZsq{}}\PY{l+s+s1}{date\PYZus{}time}\PY{l+s+s1}{\PYZsq{}}\PY{p}{]}\PY{o}{.}\PY{n}{dt}\PY{o}{.}\PY{n}{date}
    \PY{c+c1}{\PYZsh{} Getting the time}
    \PY{n}{df}\PY{p}{[}\PY{l+s+s1}{\PYZsq{}}\PY{l+s+s1}{time}\PY{l+s+s1}{\PYZsq{}}\PY{p}{]}  \PY{o}{=} \PY{n}{df}\PY{p}{[}\PY{l+s+s1}{\PYZsq{}}\PY{l+s+s1}{date\PYZus{}time}\PY{l+s+s1}{\PYZsq{}}\PY{p}{]}\PY{o}{.}\PY{n}{dt}\PY{o}{.}\PY{n}{time}
    \PY{c+c1}{\PYZsh{} Getting the ymonth}
    \PY{n}{df}\PY{p}{[}\PY{l+s+s1}{\PYZsq{}}\PY{l+s+s1}{ymonth}\PY{l+s+s1}{\PYZsq{}}\PY{p}{]} \PY{o}{=} \PY{n}{df}\PY{p}{[}\PY{l+s+s1}{\PYZsq{}}\PY{l+s+s1}{date\PYZus{}time}\PY{l+s+s1}{\PYZsq{}}\PY{p}{]}\PY{o}{.}\PY{n}{apply}\PY{p}{(}\PY{k}{lambda} \PY{n}{x}\PY{p}{:}\PY{n}{x}\PY{o}{.}\PY{n}{strftime}\PY{p}{(}\PY{l+s+s1}{\PYZsq{}}\PY{l+s+s1}{\PYZpc{}}\PY{l+s+s1}{Y\PYZhy{}}\PY{l+s+s1}{\PYZpc{}}\PY{l+s+s1}{m}\PY{l+s+s1}{\PYZsq{}}\PY{p}{)}\PY{p}{)}
    \PY{c+c1}{\PYZsh{} Getting the month}
    \PY{n}{df}\PY{p}{[}\PY{l+s+s1}{\PYZsq{}}\PY{l+s+s1}{month}\PY{l+s+s1}{\PYZsq{}}\PY{p}{]} \PY{o}{=} \PY{n}{df}\PY{p}{[}\PY{l+s+s1}{\PYZsq{}}\PY{l+s+s1}{date\PYZus{}time}\PY{l+s+s1}{\PYZsq{}}\PY{p}{]}\PY{o}{.}\PY{n}{dt}\PY{o}{.}\PY{n}{month}
    \PY{c+c1}{\PYZsh{} Replacing the month with text.}
    \PY{n}{df}\PY{p}{[}\PY{l+s+s1}{\PYZsq{}}\PY{l+s+s1}{month}\PY{l+s+s1}{\PYZsq{}}\PY{p}{]} \PY{o}{=} \PY{n}{df}\PY{p}{[}\PY{l+s+s1}{\PYZsq{}}\PY{l+s+s1}{month}\PY{l+s+s1}{\PYZsq{}}\PY{p}{]}\PY{o}{.}\PY{n}{replace}\PY{p}{(}\PY{p}{(}\PY{l+m+mi}{1}\PY{p}{,} \PY{l+m+mi}{2}\PY{p}{,} \PY{l+m+mi}{3}\PY{p}{,} \PY{l+m+mi}{4}\PY{p}{,} \PY{l+m+mi}{5}\PY{p}{,} \PY{l+m+mi}{6}\PY{p}{,} \PY{l+m+mi}{7}\PY{p}{,} \PY{l+m+mi}{8}\PY{p}{,} \PY{l+m+mi}{9}\PY{p}{,} \PY{l+m+mi}{10}\PY{p}{,}\PY{l+m+mi}{11}\PY{p}{,} \PY{l+m+mi}{12}\PY{p}{)}\PY{p}{,} 
             \PY{p}{(}\PY{l+s+s1}{\PYZsq{}}\PY{l+s+s1}{January}\PY{l+s+s1}{\PYZsq{}}\PY{p}{,}\PY{l+s+s1}{\PYZsq{}}\PY{l+s+s1}{February}\PY{l+s+s1}{\PYZsq{}}\PY{p}{,} \PY{l+s+s1}{\PYZsq{}}\PY{l+s+s1}{March}\PY{l+s+s1}{\PYZsq{}}\PY{p}{,} \PY{l+s+s1}{\PYZsq{}}\PY{l+s+s1}{April}\PY{l+s+s1}{\PYZsq{}}\PY{p}{,} \PY{l+s+s1}{\PYZsq{}}\PY{l+s+s1}{May}\PY{l+s+s1}{\PYZsq{}}\PY{p}{,} \PY{l+s+s1}{\PYZsq{}}\PY{l+s+s1}{June}\PY{l+s+s1}{\PYZsq{}}\PY{p}{,} \PY{l+s+s1}{\PYZsq{}}\PY{l+s+s1}{July}\PY{l+s+s1}{\PYZsq{}}\PY{p}{,} \PY{l+s+s1}{\PYZsq{}}\PY{l+s+s1}{August}\PY{l+s+s1}{\PYZsq{}}\PY{p}{,} \PY{l+s+s1}{\PYZsq{}}\PY{l+s+s1}{September}\PY{l+s+s1}{\PYZsq{}}\PY{p}{,} 
              \PY{l+s+s1}{\PYZsq{}}\PY{l+s+s1}{October}\PY{l+s+s1}{\PYZsq{}}\PY{p}{,} \PY{l+s+s1}{\PYZsq{}}\PY{l+s+s1}{November}\PY{l+s+s1}{\PYZsq{}}\PY{p}{,} \PY{l+s+s1}{\PYZsq{}}\PY{l+s+s1}{December}\PY{l+s+s1}{\PYZsq{}}\PY{p}{)}\PY{p}{)}
    \PY{c+c1}{\PYZsh{} Getting the weekday}
    \PY{n}{df}\PY{p}{[}\PY{l+s+s1}{\PYZsq{}}\PY{l+s+s1}{weekday}\PY{l+s+s1}{\PYZsq{}}\PY{p}{]} \PY{o}{=} \PY{n}{df}\PY{p}{[}\PY{l+s+s1}{\PYZsq{}}\PY{l+s+s1}{date\PYZus{}time}\PY{l+s+s1}{\PYZsq{}}\PY{p}{]}\PY{o}{.}\PY{n}{dt}\PY{o}{.}\PY{n}{weekday}
    \PY{c+c1}{\PYZsh{} Replacing the week with text.}
    \PY{n}{df}\PY{p}{[}\PY{l+s+s1}{\PYZsq{}}\PY{l+s+s1}{weekday}\PY{l+s+s1}{\PYZsq{}}\PY{p}{]} \PY{o}{=} \PY{n}{df}\PY{p}{[}\PY{l+s+s1}{\PYZsq{}}\PY{l+s+s1}{weekday}\PY{l+s+s1}{\PYZsq{}}\PY{p}{]}\PY{o}{.}\PY{n}{replace}\PY{p}{(}\PY{p}{(}\PY{l+m+mi}{0}\PY{p}{,}\PY{l+m+mi}{1}\PY{p}{,}\PY{l+m+mi}{2}\PY{p}{,}\PY{l+m+mi}{3}\PY{p}{,}\PY{l+m+mi}{4}\PY{p}{,}\PY{l+m+mi}{5}\PY{p}{,}\PY{l+m+mi}{6}\PY{p}{)}\PY{p}{,} 
                                          \PY{p}{(}\PY{l+s+s1}{\PYZsq{}}\PY{l+s+s1}{Monday}\PY{l+s+s1}{\PYZsq{}}\PY{p}{,}\PY{l+s+s1}{\PYZsq{}}\PY{l+s+s1}{Tuesday}\PY{l+s+s1}{\PYZsq{}}\PY{p}{,}\PY{l+s+s1}{\PYZsq{}}\PY{l+s+s1}{Wednesday}\PY{l+s+s1}{\PYZsq{}}\PY{p}{,}\PY{l+s+s1}{\PYZsq{}}\PY{l+s+s1}{Thursday}\PY{l+s+s1}{\PYZsq{}}\PY{p}{,}\PY{l+s+s1}{\PYZsq{}}\PY{l+s+s1}{Friday}\PY{l+s+s1}{\PYZsq{}}\PY{p}{,}\PY{l+s+s1}{\PYZsq{}}\PY{l+s+s1}{Saturday}\PY{l+s+s1}{\PYZsq{}}\PY{p}{,}\PY{l+s+s1}{\PYZsq{}}\PY{l+s+s1}{Sunday}\PY{l+s+s1}{\PYZsq{}}\PY{p}{)}\PY{p}{)}
    \PY{c+c1}{\PYZsh{} Getting the hour}
    \PY{n}{df}\PY{p}{[}\PY{l+s+s1}{\PYZsq{}}\PY{l+s+s1}{rhour}\PY{l+s+s1}{\PYZsq{}}\PY{p}{]} \PY{o}{=} \PY{n}{df}\PY{p}{[}\PY{l+s+s1}{\PYZsq{}}\PY{l+s+s1}{date\PYZus{}time}\PY{l+s+s1}{\PYZsq{}}\PY{p}{]}\PY{o}{.}\PY{n}{dt}\PY{o}{.}\PY{n}{hour}
    \PY{c+c1}{\PYZsh{} Replacing the hour with different time range text.}
    \PY{n}{df}\PY{p}{[}\PY{l+s+s1}{\PYZsq{}}\PY{l+s+s1}{rhour}\PY{l+s+s1}{\PYZsq{}}\PY{p}{]} \PY{o}{=} \PY{n}{df}\PY{p}{[}\PY{l+s+s1}{\PYZsq{}}\PY{l+s+s1}{rhour}\PY{l+s+s1}{\PYZsq{}}\PY{p}{]}\PY{o}{.}\PY{n}{replace}\PY{p}{(}\PY{p}{(}\PY{l+m+mi}{1}\PY{p}{,} \PY{l+m+mi}{2}\PY{p}{,} \PY{l+m+mi}{3}\PY{p}{,} \PY{l+m+mi}{4}\PY{p}{,} \PY{l+m+mi}{5}\PY{p}{,} \PY{l+m+mi}{6}\PY{p}{,} \PY{l+m+mi}{7}\PY{p}{,} \PY{l+m+mi}{8}\PY{p}{,} \PY{l+m+mi}{9}\PY{p}{,} \PY{l+m+mi}{10}\PY{p}{,} \PY{l+m+mi}{11}\PY{p}{,} \PY{l+m+mi}{12}\PY{p}{,} \PY{l+m+mi}{13}\PY{p}{,} \PY{l+m+mi}{14}\PY{p}{,} \PY{l+m+mi}{15}\PY{p}{,} \PY{l+m+mi}{16}\PY{p}{,} \PY{l+m+mi}{17}\PY{p}{,}
                                       \PY{l+m+mi}{18}\PY{p}{,} \PY{l+m+mi}{19}\PY{p}{,} \PY{l+m+mi}{20}\PY{p}{,} \PY{l+m+mi}{21}\PY{p}{,} \PY{l+m+mi}{22}\PY{p}{,} \PY{l+m+mi}{23}\PY{p}{,}\PY{l+m+mi}{0}\PY{p}{)}\PY{p}{,}
                                       \PY{p}{(}\PY{l+s+s1}{\PYZsq{}}\PY{l+s+s1}{00\PYZhy{}01}\PY{l+s+s1}{\PYZsq{}}\PY{p}{,} \PY{l+s+s1}{\PYZsq{}}\PY{l+s+s1}{01\PYZhy{}02}\PY{l+s+s1}{\PYZsq{}}\PY{p}{,} \PY{l+s+s1}{\PYZsq{}}\PY{l+s+s1}{02\PYZhy{}03}\PY{l+s+s1}{\PYZsq{}}\PY{p}{,} \PY{l+s+s1}{\PYZsq{}}\PY{l+s+s1}{03\PYZhy{}04}\PY{l+s+s1}{\PYZsq{}}\PY{p}{,} \PY{l+s+s1}{\PYZsq{}}\PY{l+s+s1}{04\PYZhy{}05}\PY{l+s+s1}{\PYZsq{}}\PY{p}{,} \PY{l+s+s1}{\PYZsq{}}\PY{l+s+s1}{05\PYZhy{}06}\PY{l+s+s1}{\PYZsq{}}\PY{p}{,} \PY{l+s+s1}{\PYZsq{}}\PY{l+s+s1}{06\PYZhy{}07}\PY{l+s+s1}{\PYZsq{}}\PY{p}{,} \PY{l+s+s1}{\PYZsq{}}\PY{l+s+s1}{07\PYZhy{}08}\PY{l+s+s1}{\PYZsq{}}\PY{p}{,} 
                                        \PY{l+s+s1}{\PYZsq{}}\PY{l+s+s1}{08\PYZhy{}09}\PY{l+s+s1}{\PYZsq{}}\PY{p}{,} \PY{l+s+s1}{\PYZsq{}}\PY{l+s+s1}{09\PYZhy{}10}\PY{l+s+s1}{\PYZsq{}}\PY{p}{,} \PY{l+s+s1}{\PYZsq{}}\PY{l+s+s1}{10\PYZhy{}11}\PY{l+s+s1}{\PYZsq{}}\PY{p}{,} \PY{l+s+s1}{\PYZsq{}}\PY{l+s+s1}{11\PYZhy{}12}\PY{l+s+s1}{\PYZsq{}}\PY{p}{,} \PY{l+s+s1}{\PYZsq{}}\PY{l+s+s1}{12\PYZhy{}13}\PY{l+s+s1}{\PYZsq{}}\PY{p}{,} \PY{l+s+s1}{\PYZsq{}}\PY{l+s+s1}{13\PYZhy{}14}\PY{l+s+s1}{\PYZsq{}}\PY{p}{,} \PY{l+s+s1}{\PYZsq{}}\PY{l+s+s1}{14\PYZhy{}15}\PY{l+s+s1}{\PYZsq{}}\PY{p}{,} \PY{l+s+s1}{\PYZsq{}}\PY{l+s+s1}{15\PYZhy{}16}\PY{l+s+s1}{\PYZsq{}}\PY{p}{,} 
                                        \PY{l+s+s1}{\PYZsq{}}\PY{l+s+s1}{16\PYZhy{}17}\PY{l+s+s1}{\PYZsq{}}\PY{p}{,} \PY{l+s+s1}{\PYZsq{}}\PY{l+s+s1}{17\PYZhy{}18}\PY{l+s+s1}{\PYZsq{}}\PY{p}{,} \PY{l+s+s1}{\PYZsq{}}\PY{l+s+s1}{18\PYZhy{}19}\PY{l+s+s1}{\PYZsq{}}\PY{p}{,} \PY{l+s+s1}{\PYZsq{}}\PY{l+s+s1}{19\PYZhy{}20}\PY{l+s+s1}{\PYZsq{}}\PY{p}{,} \PY{l+s+s1}{\PYZsq{}}\PY{l+s+s1}{20\PYZhy{}21}\PY{l+s+s1}{\PYZsq{}}\PY{p}{,} \PY{l+s+s1}{\PYZsq{}}\PY{l+s+s1}{21\PYZhy{}22}\PY{l+s+s1}{\PYZsq{}}\PY{p}{,} \PY{l+s+s1}{\PYZsq{}}\PY{l+s+s1}{22\PYZhy{}23}\PY{l+s+s1}{\PYZsq{}}\PY{p}{,} \PY{l+s+s1}{\PYZsq{}}\PY{l+s+s1}{23\PYZhy{}00}\PY{l+s+s1}{\PYZsq{}}\PY{p}{)}\PY{p}{)}
    \PY{c+c1}{\PYZsh{} add a new column}
    \PY{n}{df}\PY{p}{[}\PY{l+s+s1}{\PYZsq{}}\PY{l+s+s1}{num}\PY{l+s+s1}{\PYZsq{}}\PY{p}{]} \PY{o}{=} \PY{l+m+mi}{1}
    
    \PY{k}{return} \PY{n}{df}

\PY{n}{df} \PY{o}{=} \PY{n}{dataset\PYZus{}datetime\PYZus{}split}\PY{p}{(}\PY{n}{df}\PY{p}{)}
\PY{n}{df}\PY{o}{.}\PY{n}{head}\PY{p}{(}\PY{l+m+mi}{10}\PY{p}{)}
\end{Verbatim}
\end{tcolorbox}

            \begin{tcolorbox}[breakable, size=fbox, boxrule=.5pt, pad at break*=1mm, opacityfill=0]
\prompt{Out}{outcolor}{10}{\boxspacing}
\begin{Verbatim}[commandchars=\\\{\}]
   Transaction           Item           date\_time period\_day weekday\_weekend  \textbackslash{}
0            1          Bread 2016-10-30 09:58:00    morning         weekend
1            2   Scandinavian 2016-10-30 10:05:00    morning         weekend
2            2   Scandinavian 2016-10-30 10:05:00    morning         weekend
3            3  Hot chocolate 2016-10-30 10:07:00    morning         weekend
4            3            Jam 2016-10-30 10:07:00    morning         weekend
5            3        Cookies 2016-10-30 10:07:00    morning         weekend
6            4         Muffin 2016-10-30 10:08:00    morning         weekend
7            5         Coffee 2016-10-30 10:13:00    morning         weekend
8            5         Pastry 2016-10-30 10:13:00    morning         weekend
9            5          Bread 2016-10-30 10:13:00    morning         weekend

         date      time   ymonth    month weekday  rhour  num
0  2016-10-30  09:58:00  2016-10  October  Sunday  08-09    1
1  2016-10-30  10:05:00  2016-10  October  Sunday  09-10    1
2  2016-10-30  10:05:00  2016-10  October  Sunday  09-10    1
3  2016-10-30  10:07:00  2016-10  October  Sunday  09-10    1
4  2016-10-30  10:07:00  2016-10  October  Sunday  09-10    1
5  2016-10-30  10:07:00  2016-10  October  Sunday  09-10    1
6  2016-10-30  10:08:00  2016-10  October  Sunday  09-10    1
7  2016-10-30  10:13:00  2016-10  October  Sunday  09-10    1
8  2016-10-30  10:13:00  2016-10  October  Sunday  09-10    1
9  2016-10-30  10:13:00  2016-10  October  Sunday  09-10    1
\end{Verbatim}
\end{tcolorbox}
        
    The following function \texttt{daily\_sales()} and
\texttt{monthly\_sales()} is countting the sales respectively.

    \begin{tcolorbox}[breakable, size=fbox, boxrule=1pt, pad at break*=1mm,colback=cellbackground, colframe=cellborder]
\prompt{In}{incolor}{15}{\boxspacing}
\begin{Verbatim}[commandchars=\\\{\}]
\PY{c+c1}{\PYZsh{} Count the sales of every day.}
\PY{k}{def} \PY{n+nf}{daily\PYZus{}sales}\PY{p}{(}\PY{p}{)}\PY{p}{:}
    \PY{n}{date\PYZus{}list} \PY{o}{=} \PY{n+nb}{list}\PY{p}{(}\PY{n}{df}\PY{p}{[}\PY{l+s+s1}{\PYZsq{}}\PY{l+s+s1}{date}\PY{l+s+s1}{\PYZsq{}}\PY{p}{]}\PY{o}{.}\PY{n}{apply}\PY{p}{(}\PY{k}{lambda} \PY{n}{x}\PY{p}{:}\PY{n}{x}\PY{o}{.}\PY{n}{strftime}\PY{p}{(}\PY{l+s+s1}{\PYZsq{}}\PY{l+s+s1}{\PYZpc{}}\PY{l+s+s1}{Y\PYZhy{}}\PY{l+s+s1}{\PYZpc{}}\PY{l+s+s1}{m\PYZhy{}}\PY{l+s+si}{\PYZpc{}d}\PY{l+s+s1}{\PYZsq{}}\PY{p}{)}\PY{p}{)}\PY{p}{)}  
    \PY{n}{trans\PYZus{}dates} \PY{o}{=} \PY{p}{[}\PY{p}{]}
    \PY{k}{for} \PY{n}{d} \PY{o+ow}{in} \PY{n}{date\PYZus{}list}\PY{p}{:}
        \PY{k}{if} \PY{n}{d} \PY{o+ow}{not} \PY{o+ow}{in} \PY{n}{trans\PYZus{}dates}\PY{p}{:}
            \PY{n}{trans\PYZus{}dates}\PY{o}{.}\PY{n}{append}\PY{p}{(}\PY{n}{d}\PY{p}{)}
    \PY{n}{sales\PYZus{}num} \PY{o}{=} \PY{p}{[}\PY{p}{]}       
    \PY{k}{for} \PY{n}{t} \PY{o+ow}{in} \PY{n}{trans\PYZus{}dates}\PY{p}{:}
        \PY{n}{num} \PY{o}{=} \PY{l+m+mi}{0}
        \PY{k}{for} \PY{n}{d} \PY{o+ow}{in} \PY{n}{date\PYZus{}list}\PY{p}{:}
            \PY{k}{if} \PY{n}{t} \PY{o}{==} \PY{n}{d}\PY{p}{:}
                \PY{n}{num} \PY{o}{+}\PY{o}{=}\PY{l+m+mi}{1}
        \PY{n}{sales\PYZus{}num}\PY{o}{.}\PY{n}{append}\PY{p}{(}\PY{n}{num}\PY{p}{)}
        
    \PY{k}{return} \PY{n}{trans\PYZus{}dates}\PY{p}{,} \PY{n}{sales\PYZus{}num}

\PY{c+c1}{\PYZsh{} Count the sales of every month}
\PY{k}{def} \PY{n+nf}{monthly\PYZus{}sales}\PY{p}{(}\PY{p}{)}\PY{p}{:}
    \PY{n}{ymonth\PYZus{}list} \PY{o}{=} \PY{n+nb}{list}\PY{p}{(}\PY{n}{df}\PY{p}{[}\PY{l+s+s1}{\PYZsq{}}\PY{l+s+s1}{ymonth}\PY{l+s+s1}{\PYZsq{}}\PY{p}{]}\PY{p}{)}   
    \PY{n}{trans\PYZus{}months} \PY{o}{=} \PY{p}{[}\PY{p}{]}
    \PY{k}{for} \PY{n}{d} \PY{o+ow}{in} \PY{n}{ymonth\PYZus{}list}\PY{p}{:}
        \PY{k}{if} \PY{n}{d} \PY{o+ow}{not} \PY{o+ow}{in} \PY{n}{trans\PYZus{}months}\PY{p}{:}
            \PY{n}{trans\PYZus{}months}\PY{o}{.}\PY{n}{append}\PY{p}{(}\PY{n}{d}\PY{p}{)}
    \PY{n}{sales\PYZus{}num} \PY{o}{=} \PY{p}{[}\PY{p}{]}       
    \PY{k}{for} \PY{n}{t} \PY{o+ow}{in} \PY{n}{trans\PYZus{}months}\PY{p}{:}
        \PY{n}{num} \PY{o}{=} \PY{l+m+mi}{0}
        \PY{k}{for} \PY{n}{d} \PY{o+ow}{in} \PY{n}{ymonth\PYZus{}list}\PY{p}{:}
            \PY{k}{if} \PY{n}{t} \PY{o}{==} \PY{n}{d}\PY{p}{:}
                \PY{n}{num} \PY{o}{+}\PY{o}{=}\PY{l+m+mi}{1}
        \PY{n}{sales\PYZus{}num}\PY{o}{.}\PY{n}{append}\PY{p}{(}\PY{n}{num}\PY{p}{)}
        
    \PY{k}{return} \PY{n}{trans\PYZus{}months}\PY{p}{,} \PY{n}{sales\PYZus{}num}
\end{Verbatim}
\end{tcolorbox}

    The function \texttt{top\_n\_popular\_items()} is to implement the
function which will return different results based on the different
parmeters \texttt{n} and \texttt{date\_range}. For example, if the
arguments are `5' and ``{[}`09-10-2016', `11-03-2017'{]}'', the funciton
will return the top 5 popular intems by coustomers liked between
09-10-2016 and 11-03-2017.

    \begin{tcolorbox}[breakable, size=fbox, boxrule=1pt, pad at break*=1mm,colback=cellbackground, colframe=cellborder]
\prompt{In}{incolor}{70}{\boxspacing}
\begin{Verbatim}[commandchars=\\\{\}]
\PY{k+kn}{import} \PY{n+nn}{datetime}
\PY{c+c1}{\PYZsh{} Count the total sales of the top n items in period of time.}
\PY{k}{def} \PY{n+nf}{top\PYZus{}n\PYZus{}popular\PYZus{}items}\PY{p}{(}\PY{n}{n}\PY{p}{,} \PY{n}{date\PYZus{}range}\PY{p}{,} \PY{n}{df}\PY{p}{)}\PY{p}{:}
    \PY{n}{start\PYZus{}date} \PY{o}{=} \PY{n}{datetime}\PY{o}{.}\PY{n}{datetime}\PY{o}{.}\PY{n}{strptime}\PY{p}{(}\PY{n}{date\PYZus{}range}\PY{p}{[}\PY{l+m+mi}{0}\PY{p}{]}\PY{p}{,} \PY{l+s+s1}{\PYZsq{}}\PY{l+s+si}{\PYZpc{}d}\PY{l+s+s1}{\PYZhy{}}\PY{l+s+s1}{\PYZpc{}}\PY{l+s+s1}{m\PYZhy{}}\PY{l+s+s1}{\PYZpc{}}\PY{l+s+s1}{Y}\PY{l+s+s1}{\PYZsq{}}\PY{p}{)}\PY{o}{.}\PY{n}{date}\PY{p}{(}\PY{p}{)}
    \PY{n}{end\PYZus{}date} \PY{o}{=} \PY{n}{datetime}\PY{o}{.}\PY{n}{datetime}\PY{o}{.}\PY{n}{strptime}\PY{p}{(}\PY{n}{date\PYZus{}range}\PY{p}{[}\PY{l+m+mi}{1}\PY{p}{]}\PY{p}{,} \PY{l+s+s1}{\PYZsq{}}\PY{l+s+si}{\PYZpc{}d}\PY{l+s+s1}{\PYZhy{}}\PY{l+s+s1}{\PYZpc{}}\PY{l+s+s1}{m\PYZhy{}}\PY{l+s+s1}{\PYZpc{}}\PY{l+s+s1}{Y}\PY{l+s+s1}{\PYZsq{}}\PY{p}{)}\PY{o}{.}\PY{n}{date}\PY{p}{(}\PY{p}{)}
    \PY{n}{df} \PY{o}{=} \PY{n}{df}\PY{p}{[}\PY{p}{(}\PY{n}{df}\PY{p}{[}\PY{l+s+s1}{\PYZsq{}}\PY{l+s+s1}{date}\PY{l+s+s1}{\PYZsq{}}\PY{p}{]} \PY{o}{\PYZgt{}}\PY{o}{=} \PY{n}{start\PYZus{}date}\PY{p}{)} \PY{o}{\PYZam{}} \PY{p}{(}\PY{n}{df}\PY{p}{[}\PY{l+s+s1}{\PYZsq{}}\PY{l+s+s1}{date}\PY{l+s+s1}{\PYZsq{}}\PY{p}{]} \PY{o}{\PYZlt{}}\PY{o}{=} \PY{n}{end\PYZus{}date}\PY{p}{)}\PY{p}{]}
    \PY{n}{x} \PY{o}{=} \PY{n}{df}\PY{o}{.}\PY{n}{Item}\PY{o}{.}\PY{n}{value\PYZus{}counts}\PY{p}{(}\PY{p}{)}\PY{o}{.}\PY{n}{head}\PY{p}{(}\PY{n}{n}\PY{p}{)}\PY{o}{.}\PY{n}{index}
    \PY{n}{y} \PY{o}{=} \PY{n}{df}\PY{o}{.}\PY{n}{Item}\PY{o}{.}\PY{n}{value\PYZus{}counts}\PY{p}{(}\PY{p}{)}\PY{o}{.}\PY{n}{head}\PY{p}{(}\PY{n}{n}\PY{p}{)}\PY{o}{.}\PY{n}{values}
    
    \PY{k}{return} \PY{n}{x}\PY{p}{,} \PY{n}{y}
\end{Verbatim}
\end{tcolorbox}

    The function \texttt{apriori\_model\_algorithm()} is a simple algorithm
based on the method of association rule learning. Association Rules
reflect the interdependence and association between a thing and other
things. It is an important technology of data mining, which is used to
mine the correlation between valuable data items from a large amount of
data.In this function, we use Apriori algorithm to implement the
correlation between different items. Finally, when a customer purchases
a food item, we can recommend a item to this customer based on the
correlation rules.

    \begin{tcolorbox}[breakable, size=fbox, boxrule=1pt, pad at break*=1mm,colback=cellbackground, colframe=cellborder]
\prompt{In}{incolor}{18}{\boxspacing}
\begin{Verbatim}[commandchars=\\\{\}]
\PY{k+kn}{from} \PY{n+nn}{mlxtend}\PY{n+nn}{.}\PY{n+nn}{frequent\PYZus{}patterns} \PY{k}{import} \PY{n}{association\PYZus{}rules}\PY{p}{,} \PY{n}{apriori}
\PY{k+kn}{from} \PY{n+nn}{mlxtend}\PY{n+nn}{.}\PY{n+nn}{preprocessing} \PY{k}{import} \PY{n}{TransactionEncoder}

\PY{n}{trans\PYZus{}items} \PY{o}{=} \PY{p}{[}\PY{p}{]}
\PY{k}{for} \PY{n}{i} \PY{o+ow}{in} \PY{n+nb}{range}\PY{p}{(}\PY{l+m+mi}{1}\PY{p}{,} \PY{n}{df}\PY{p}{[}\PY{l+s+s1}{\PYZsq{}}\PY{l+s+s1}{Transaction}\PY{l+s+s1}{\PYZsq{}}\PY{p}{]}\PY{o}{.}\PY{n}{nunique}\PY{p}{(}\PY{p}{)} \PY{o}{+} \PY{l+m+mi}{1}\PY{p}{)}\PY{p}{:}
    \PY{n}{temp} \PY{o}{=} \PY{n}{df}\PY{p}{[}\PY{n}{df}\PY{p}{[}\PY{l+s+s1}{\PYZsq{}}\PY{l+s+s1}{Transaction}\PY{l+s+s1}{\PYZsq{}}\PY{p}{]} \PY{o}{==} \PY{n}{i}\PY{p}{]}
    \PY{n}{items} \PY{o}{=} \PY{n+nb}{list}\PY{p}{(}\PY{n}{temp}\PY{p}{[}\PY{l+s+s1}{\PYZsq{}}\PY{l+s+s1}{Item}\PY{l+s+s1}{\PYZsq{}}\PY{p}{]}\PY{p}{)}
    \PY{k}{if} \PY{p}{(}\PY{n+nb}{len}\PY{p}{(}\PY{n}{items}\PY{p}{)} \PY{o}{\PYZgt{}} \PY{l+m+mi}{0}\PY{p}{)}\PY{p}{:}
        \PY{n}{trans\PYZus{}items}\PY{o}{.}\PY{n}{append}\PY{p}{(}\PY{n}{items}\PY{p}{)}
\end{Verbatim}
\end{tcolorbox}

    \begin{tcolorbox}[breakable, size=fbox, boxrule=1pt, pad at break*=1mm,colback=cellbackground, colframe=cellborder]
\prompt{In}{incolor}{105}{\boxspacing}
\begin{Verbatim}[commandchars=\\\{\}]
\PY{k}{def} \PY{n+nf}{apriori\PYZus{}model\PYZus{}algorithm}\PY{p}{(}\PY{n}{trans\PYZus{}items}\PY{p}{)}\PY{p}{:}
    \PY{n}{encoder} \PY{o}{=} \PY{n}{TransactionEncoder}\PY{p}{(}\PY{p}{)}
    \PY{c+c1}{\PYZsh{} one\PYZhy{}hot encode}
    \PY{n}{te\PYZus{}ary} \PY{o}{=} \PY{n}{encoder}\PY{o}{.}\PY{n}{fit}\PY{p}{(}\PY{n}{trans\PYZus{}items}\PY{p}{)}\PY{o}{.}\PY{n}{transform}\PY{p}{(}\PY{n}{trans\PYZus{}items}\PY{p}{)}
    \PY{n}{df} \PY{o}{=} \PY{n}{pd}\PY{o}{.}\PY{n}{DataFrame}\PY{p}{(}\PY{n}{te\PYZus{}ary}\PY{p}{,} \PY{n}{columns}\PY{o}{=}\PY{n}{encoder}\PY{o}{.}\PY{n}{columns\PYZus{}}\PY{p}{)}
    \PY{c+c1}{\PYZsh{} Find the frequent items}
    \PY{n}{frequent\PYZus{}items} \PY{o}{=} \PY{n}{apriori}\PY{p}{(}\PY{n}{df}\PY{p}{,} \PY{n}{min\PYZus{}support}\PY{o}{=}\PY{l+m+mf}{0.01}\PY{p}{,} \PY{n}{use\PYZus{}colnames}\PY{o}{=}\PY{k+kc}{True}\PY{p}{)}
    
    \PY{n}{result} \PY{o}{=} \PY{n}{association\PYZus{}rules}\PY{p}{(}\PY{n}{frequent\PYZus{}items}\PY{p}{,} \PY{n}{metric}\PY{o}{=}\PY{l+s+s2}{\PYZdq{}}\PY{l+s+s2}{confidence}\PY{l+s+s2}{\PYZdq{}}\PY{p}{,} \PY{n}{min\PYZus{}threshold}\PY{o}{=}\PY{l+m+mf}{0.3}\PY{p}{)}
    \PY{n}{result}\PY{o}{.}\PY{n}{sort\PYZus{}values}\PY{p}{(}\PY{l+s+s1}{\PYZsq{}}\PY{l+s+s1}{confidence}\PY{l+s+s1}{\PYZsq{}}\PY{p}{,} \PY{n}{ascending}\PY{o}{=}\PY{k+kc}{False}\PY{p}{,} \PY{n}{inplace}\PY{o}{=}\PY{k+kc}{True}\PY{p}{)}
    
    \PY{k}{return} \PY{n}{result}

\PY{n}{display}\PY{p}{(}\PY{n}{apriori\PYZus{}model\PYZus{}algorithm}\PY{p}{(}\PY{n}{trans\PYZus{}items}\PY{p}{)}\PY{p}{)}
\end{Verbatim}
\end{tcolorbox}

    
    \begin{verbatim}
         antecedents consequents  antecedent support  consequent support  \
15           (Toast)    (Coffee)            0.033272            0.478881   
13  (Spanish Brunch)    (Coffee)            0.017608            0.478881   
7        (Medialuna)    (Coffee)            0.062763            0.478881   
9           (Pastry)    (Coffee)            0.086745            0.478881   
6            (Juice)    (Coffee)            0.037809            0.478881   
0        (Alfajores)    (Coffee)            0.035757            0.478881   
10        (Sandwich)    (Coffee)            0.071081            0.478881   
3             (Cake)    (Coffee)            0.103813            0.478881   
11           (Scone)    (Coffee)            0.034784            0.478881   
4          (Cookies)    (Coffee)            0.054445            0.478881   
5    (Hot chocolate)    (Coffee)            0.059198            0.478881   
2          (Brownie)    (Coffee)            0.040402            0.478881   
8           (Muffin)    (Coffee)            0.038025            0.478881   
12            (Soup)    (Coffee)            0.034352            0.478881   
16     (Bread, Cake)    (Coffee)            0.023334            0.478881   
18       (Cake, Tea)    (Coffee)            0.023658            0.478881   
17   (Bread, Pastry)    (Coffee)            0.029275            0.478881   
14             (Tea)    (Coffee)            0.142811            0.478881   
1           (Pastry)     (Bread)            0.086745            0.328184   

     support  confidence      lift  leverage  conviction  
15  0.023226    0.698052  1.457674  0.007292    1.725857  
13  0.010479    0.595092  1.242672  0.002046    1.287006  
7   0.035541    0.566265  1.182476  0.005485    1.201469  
9   0.047964    0.552927  1.154622  0.006423    1.165623  
6   0.020417    0.540000  1.127629  0.002311    1.132868  
0   0.019229    0.537764  1.122961  0.002105    1.127388  
10  0.037917    0.533435  1.113919  0.003878    1.116926  
3   0.054769    0.527575  1.101684  0.005055    1.103074  
11  0.018040    0.518634  1.083012  0.001383    1.082583  
4   0.028087    0.515873  1.077247  0.002014    1.076410  
5   0.029923    0.505474  1.055533  0.001574    1.053776  
2   0.019877    0.491979  1.027351  0.000529    1.025782  
8   0.018689    0.491477  1.026304  0.000479    1.024771  
12  0.015880    0.462264  0.965301 -0.000571    0.969099  
16  0.010154    0.435185  0.908755 -0.001020    0.922637  
18  0.010046    0.424658  0.886771 -0.001283    0.905755  
17  0.011235    0.383764  0.801376 -0.002785    0.845648  
14  0.050232    0.351740  0.734504 -0.018157    0.803873  
1   0.029275    0.337484  1.028339  0.000807    1.014038  
    \end{verbatim}

    
    \hypertarget{project-outcome-10-10-marks}{%
\section{Project Outcome (10 + 10
marks)}\label{project-outcome-10-10-marks}}

    \hypertarget{overview-of-results}{%
\subsection{Overview of Results}\label{overview-of-results}}

This is a general overview of the results. The first objective is
counting the historical sales volume. We will show the results through
two line charts(Figure1 and Figure2). Figure one is about the sales
trend of every day. Figure two is about the sales trend of every month.
We can get some statistic data from the two charts.

The second objective is finding the top n popular products in a period
of time. In this objective, we will use two different charts(Figure3 and
Figure4) to show the result. Figure three is a bar chart showing in the
top ten popular items. Figure four is a pie chart showing the percentage
of every item among the top ten items. We can find the total sale of
every item and can know which item is the most popular product in this
bakery.

The last objective is recommending products to customers. We use an
unsupervised algorithm to complete this objective. We build a table
about the correlations of every item. Then, we can recommend a product
to a customer based on two indexes (confidence and lift).

For details, please see every section below.

    \hypertarget{objective-1-count-historical-sales-volumedaily-monthly}{%
\subsection{Objective 1 : Count historical sales volume(daily,
monthly)}\label{objective-1-count-historical-sales-volumedaily-monthly}}

\hypertarget{explanation-of-results}{%
\subsubsection{Explanation of Results}\label{explanation-of-results}}

This aim is to count the total sales of every day and every month in a
period of time. Figure one is about the total sales of every day. It is
plotted in a line chart. In order to plot this chart clearly, we just
select the first 50 entries. As this figure is shown below, this is a
sales trend every day. We can see the maximum sales (about 280) and
minimum() sales(about 40) from this picture, and the average sales per
day are about 150. So, we can get the statistics data from this figure.

Similarly, figure 2 is about the sales trend of every month by counting
the total sales of every month from 2016-10-30 to 2017-04-11. the total
sales of the first month and last month are lower than in other months
because the data-set is incomplete. From the second figure, we can found
the total sales of every month is around 3900 except for the first month
and the last month.

To sum up, this is an efficient solution by describing the sales trend,
which can help the bakery manager to view the sale figure easily.

\hypertarget{visualisation}{%
\subsubsection{Visualisation}\label{visualisation}}

The following line chart gives a vivid representation of trend sales of
every day and every month.

    \begin{tcolorbox}[breakable, size=fbox, boxrule=1pt, pad at break*=1mm,colback=cellbackground, colframe=cellborder]
\prompt{In}{incolor}{112}{\boxspacing}
\begin{Verbatim}[commandchars=\\\{\}]
\PY{n}{sns}\PY{o}{.}\PY{n}{set\PYZus{}theme}\PY{p}{(}\PY{n}{style}\PY{o}{=}\PY{l+s+s2}{\PYZdq{}}\PY{l+s+s2}{whitegrid}\PY{l+s+s2}{\PYZdq{}}\PY{p}{)}
\PY{c+c1}{\PYZsh{} The total sales of every day}
\PY{n}{dx}\PY{p}{,} \PY{n}{dy} \PY{o}{=} \PY{n}{daily\PYZus{}sales}\PY{p}{(}\PY{p}{)}
\PY{n}{f1} \PY{o}{=} \PY{n}{plt}\PY{o}{.}\PY{n}{figure}\PY{p}{(}\PY{n}{figsize}\PY{o}{=}\PY{p}{(}\PY{l+m+mi}{15}\PY{p}{,}\PY{l+m+mi}{5}\PY{p}{)}\PY{p}{)}
\PY{n}{sns}\PY{o}{.}\PY{n}{lineplot}\PY{p}{(}\PY{n}{dx}\PY{p}{[}\PY{l+m+mi}{0}\PY{p}{:}\PY{l+m+mi}{50}\PY{p}{]}\PY{p}{,} \PY{n}{dy}\PY{p}{[}\PY{l+m+mi}{0}\PY{p}{:}\PY{l+m+mi}{50}\PY{p}{]}\PY{p}{,} \PY{n}{palette}\PY{o}{=}\PY{l+s+s2}{\PYZdq{}}\PY{l+s+s2}{tab10}\PY{l+s+s2}{\PYZdq{}}\PY{p}{,} \PY{n}{linewidth}\PY{o}{=}\PY{l+m+mf}{1.5}\PY{p}{)}
\PY{n}{plt}\PY{o}{.}\PY{n}{xlabel}\PY{p}{(}\PY{l+s+s1}{\PYZsq{}}\PY{l+s+s1}{Days}\PY{l+s+s1}{\PYZsq{}}\PY{p}{,} \PY{n}{size} \PY{o}{=} \PY{l+m+mi}{15}\PY{p}{)}
\PY{n}{plt}\PY{o}{.}\PY{n}{xticks}\PY{p}{(}\PY{n}{rotation}\PY{o}{=}\PY{l+m+mi}{90}\PY{p}{)}
\PY{n}{plt}\PY{o}{.}\PY{n}{ylabel}\PY{p}{(}\PY{l+s+s1}{\PYZsq{}}\PY{l+s+s1}{Count of sales}\PY{l+s+s1}{\PYZsq{}}\PY{p}{,} \PY{n}{size} \PY{o}{=} \PY{l+m+mi}{15}\PY{p}{)}
\PY{n}{plt}\PY{o}{.}\PY{n}{title}\PY{p}{(}\PY{l+s+s1}{\PYZsq{}}\PY{l+s+s1}{Figure1 \PYZhy{} The sales trend of every day}\PY{l+s+s1}{\PYZsq{}}\PY{p}{,} \PY{n}{color} \PY{o}{=} \PY{l+s+s1}{\PYZsq{}}\PY{l+s+s1}{black}\PY{l+s+s1}{\PYZsq{}}\PY{p}{,} \PY{n}{size}\PY{o}{=}\PY{l+m+mi}{20}\PY{p}{)}

\PY{c+c1}{\PYZsh{} The total sales of every month}
\PY{n}{mx}\PY{p}{,} \PY{n}{my} \PY{o}{=} \PY{n}{monthly\PYZus{}sales}\PY{p}{(}\PY{p}{)}
\PY{n}{f2} \PY{o}{=} \PY{n}{plt}\PY{o}{.}\PY{n}{figure}\PY{p}{(}\PY{n}{figsize}\PY{o}{=}\PY{p}{(}\PY{l+m+mi}{15}\PY{p}{,}\PY{l+m+mi}{5}\PY{p}{)}\PY{p}{)}
\PY{n}{sns}\PY{o}{.}\PY{n}{lineplot}\PY{p}{(}\PY{n}{mx}\PY{p}{,} \PY{n}{my}\PY{p}{,} \PY{n}{palette}\PY{o}{=}\PY{l+s+s1}{\PYZsq{}}\PY{l+s+s1}{gnuplot}\PY{l+s+s1}{\PYZsq{}}\PY{p}{)}
\PY{n}{plt}\PY{o}{.}\PY{n}{xlabel}\PY{p}{(}\PY{l+s+s1}{\PYZsq{}}\PY{l+s+s1}{Months}\PY{l+s+s1}{\PYZsq{}}\PY{p}{,} \PY{n}{size} \PY{o}{=} \PY{l+m+mi}{15}\PY{p}{)}
\PY{n}{plt}\PY{o}{.}\PY{n}{ylabel}\PY{p}{(}\PY{l+s+s1}{\PYZsq{}}\PY{l+s+s1}{Count of sales}\PY{l+s+s1}{\PYZsq{}}\PY{p}{,} \PY{n}{size} \PY{o}{=} \PY{l+m+mi}{15}\PY{p}{)}
\PY{n}{plt}\PY{o}{.}\PY{n}{title}\PY{p}{(}\PY{l+s+s1}{\PYZsq{}}\PY{l+s+s1}{Figure2 \PYZhy{} The sales trend of every month}\PY{l+s+s1}{\PYZsq{}}\PY{p}{,} \PY{n}{color} \PY{o}{=} \PY{l+s+s1}{\PYZsq{}}\PY{l+s+s1}{black}\PY{l+s+s1}{\PYZsq{}}\PY{p}{,} \PY{n}{size}\PY{o}{=}\PY{l+m+mi}{20}\PY{p}{)}
\PY{n}{plt}\PY{o}{.}\PY{n}{show}\PY{p}{(}\PY{p}{)}
\end{Verbatim}
\end{tcolorbox}

    \begin{center}
    \adjustimage{max size={0.9\linewidth}{0.9\paperheight}}{A3_Data_Analysis_Project_V0_files/A3_Data_Analysis_Project_V0_19_0.png}
    \end{center}
    { \hspace*{\fill} \\}
    
    \begin{center}
    \adjustimage{max size={0.9\linewidth}{0.9\paperheight}}{A3_Data_Analysis_Project_V0_files/A3_Data_Analysis_Project_V0_19_1.png}
    \end{center}
    { \hspace*{\fill} \\}
    
    \hypertarget{objective-2-find-the-top-n-popular-products-in-a-period-of-time}{%
\subsection{Objective 2 : Find the top `n' popular products in a period
of
time}\label{objective-2-find-the-top-n-popular-products-in-a-period-of-time}}

\hypertarget{explanation-of-results}{%
\subsubsection{Explanation of Results}\label{explanation-of-results}}

In this section, we can use the function `top\_n\_popular\_items()' to
get the most liked products by customers. Figure three shown below. In
this example, we get the top ten popular items bought by customers from
2016-10-30 to 2017-04-11. As figure 3 and figure 4 shown, The coffee has
the largest sales (about 5600) among all items and it accounts for
around 36.45\% of the total sales of the top ten popular items. The
second, third and fourth items are bread, tea and Cake respectively, and
they account for around 22.15\%, 9.56\%, 6.83\% respectively.

The pastry, sandwich have similar sales (about 800) accounted for 5.20\%
among all items. Other items have different sales between 400 to 600. We
can observe that the total sales volume of the first three items
accounts for about 68.16\%.

To sum up, we can find the most popular products in this way, which can
help the bakery manager to store and prepare goods.

\hypertarget{visualisation}{%
\subsubsection{Visualisation}\label{visualisation}}

    \begin{tcolorbox}[breakable, size=fbox, boxrule=1pt, pad at break*=1mm,colback=cellbackground, colframe=cellborder]
\prompt{In}{incolor}{145}{\boxspacing}
\begin{Verbatim}[commandchars=\\\{\}]
\PY{n}{top\PYZus{}n} \PY{o}{=} \PY{l+m+mi}{10}
\PY{n}{time\PYZus{}range} \PY{o}{=} \PY{p}{[}\PY{l+s+s1}{\PYZsq{}}\PY{l+s+s1}{30\PYZhy{}10\PYZhy{}2016}\PY{l+s+s1}{\PYZsq{}}\PY{p}{,} \PY{l+s+s1}{\PYZsq{}}\PY{l+s+s1}{11\PYZhy{}04\PYZhy{}2017}\PY{l+s+s1}{\PYZsq{}}\PY{p}{]}
\PY{n}{X}\PY{p}{,} \PY{n}{Y} \PY{o}{=} \PY{n}{top\PYZus{}n\PYZus{}popular\PYZus{}items}\PY{p}{(}\PY{n}{top\PYZus{}n}\PY{p}{,} \PY{n}{time\PYZus{}range}\PY{p}{,} \PY{n}{df}\PY{p}{)}
\PY{n}{f1} \PY{o}{=} \PY{n}{plt}\PY{o}{.}\PY{n}{figure}\PY{p}{(}\PY{n}{figsize}\PY{o}{=}\PY{p}{(}\PY{l+m+mi}{15}\PY{p}{,}\PY{l+m+mi}{5}\PY{p}{)}\PY{p}{)}
\PY{n}{sns}\PY{o}{.}\PY{n}{barplot}\PY{p}{(}\PY{n}{X}\PY{p}{,} \PY{n}{Y}\PY{p}{,} \PY{n}{palette}\PY{o}{=}\PY{l+s+s2}{\PYZdq{}}\PY{l+s+s2}{tab10}\PY{l+s+s2}{\PYZdq{}}\PY{p}{,} \PY{n}{linewidth}\PY{o}{=}\PY{l+m+mf}{1.5}\PY{p}{)}
\PY{n}{plt}\PY{o}{.}\PY{n}{xlabel}\PY{p}{(}\PY{l+s+s1}{\PYZsq{}}\PY{l+s+s1}{Items}\PY{l+s+s1}{\PYZsq{}}\PY{p}{,} \PY{n}{size} \PY{o}{=} \PY{l+m+mi}{15}\PY{p}{)}
\PY{n}{plt}\PY{o}{.}\PY{n}{xticks}\PY{p}{(}\PY{n}{rotation}\PY{o}{=}\PY{l+m+mi}{30}\PY{p}{)}
\PY{n}{plt}\PY{o}{.}\PY{n}{ylabel}\PY{p}{(}\PY{l+s+s1}{\PYZsq{}}\PY{l+s+s1}{count of items}\PY{l+s+s1}{\PYZsq{}}\PY{p}{,} \PY{n}{size}\PY{o}{=}\PY{l+m+mi}{15}\PY{p}{)}
\PY{n}{title} \PY{o}{=} \PY{l+s+s1}{\PYZsq{}}\PY{l+s+s1}{Figure3 \PYZhy{} The top }\PY{l+s+s1}{\PYZsq{}} \PY{o}{+} \PY{n+nb}{str}\PY{p}{(}\PY{n}{top\PYZus{}n}\PY{p}{)} \PY{o}{+} \PY{l+s+s1}{\PYZsq{}}\PY{l+s+s1}{ popular Items}\PY{l+s+s1}{\PYZsq{}}
\PY{n}{plt}\PY{o}{.}\PY{n}{title}\PY{p}{(}\PY{n}{title}\PY{p}{,} \PY{n}{color} \PY{o}{=} \PY{l+s+s1}{\PYZsq{}}\PY{l+s+s1}{black}\PY{l+s+s1}{\PYZsq{}}\PY{p}{,} \PY{n}{size}\PY{o}{=} \PY{l+m+mi}{20}\PY{p}{)}

\PY{n}{f2} \PY{o}{=} \PY{n}{plt}\PY{o}{.}\PY{n}{figure}\PY{p}{(}\PY{n}{figsize}\PY{o}{=}\PY{p}{(}\PY{l+m+mi}{15}\PY{p}{,}\PY{l+m+mi}{7}\PY{p}{)}\PY{p}{)}
\PY{n}{plt}\PY{o}{.}\PY{n}{pie}\PY{p}{(}\PY{n}{x}\PY{o}{=}\PY{n}{Y}\PY{p}{,} \PY{n}{labels}\PY{o}{=}\PY{n}{X}\PY{p}{,} \PY{n}{autopct}\PY{o}{=}\PY{l+s+s2}{\PYZdq{}}\PY{l+s+si}{\PYZpc{}0.2f}\PY{l+s+si}{\PYZpc{}\PYZpc{}}\PY{l+s+s2}{\PYZdq{}}\PY{p}{,}\PY{n}{startangle} \PY{o}{=} \PY{l+m+mi}{90}\PY{p}{,}\PY{n}{pctdistance} \PY{o}{=} \PY{l+m+mf}{0.6}\PY{p}{,} \PY{n}{radius}\PY{o}{=}\PY{l+m+mf}{0.8}\PY{p}{)}
\PY{n}{plt}\PY{o}{.}\PY{n}{title}\PY{p}{(}\PY{l+s+s1}{\PYZsq{}}\PY{l+s+s1}{Figure4 \PYZhy{} The percentage of every items}\PY{l+s+s1}{\PYZsq{}}\PY{p}{,} \PY{n}{color} \PY{o}{=} \PY{l+s+s1}{\PYZsq{}}\PY{l+s+s1}{black}\PY{l+s+s1}{\PYZsq{}}\PY{p}{,} \PY{n}{size}\PY{o}{=} \PY{l+m+mi}{20}\PY{p}{)}
\PY{n}{plt}\PY{o}{.}\PY{n}{tight\PYZus{}layout}\PY{p}{(}\PY{p}{)}
\PY{n}{plt}\PY{o}{.}\PY{n}{show}\PY{p}{(}\PY{p}{)}
\end{Verbatim}
\end{tcolorbox}

    \begin{center}
    \adjustimage{max size={0.9\linewidth}{0.9\paperheight}}{A3_Data_Analysis_Project_V0_files/A3_Data_Analysis_Project_V0_21_0.png}
    \end{center}
    { \hspace*{\fill} \\}
    
    \begin{center}
    \adjustimage{max size={0.9\linewidth}{0.9\paperheight}}{A3_Data_Analysis_Project_V0_files/A3_Data_Analysis_Project_V0_21_1.png}
    \end{center}
    { \hspace*{\fill} \\}
    
    \hypertarget{objective-3-recommend-product-to-customers}{%
\subsection{Objective 3 : Recommend product to
customers}\label{objective-3-recommend-product-to-customers}}

\hypertarget{explanation-of-results}{%
\subsubsection{Explanation of Results}\label{explanation-of-results}}

The table is shown correlations computing by Apriori algorithm between
different items. Apriori algorithm is an unsupervised learning algorithm
and it is looking for a certain relationship between data in a bunch of
data sets the same as the K-means algorithm. In this algorithm, we need
to compute the frequent items and correlation rules. Then we need to
complete the recommendation task based on two indexes(confidence and
lift). the confidence represents the probability of a good being bought.

Int this table, for example, If we assume that a customer buys a toast
and then the probability of buying coffee is 69.80\%, and he/his maybe
buy 1.45 coffee again. Similarly, if one customer bought a pastry,
he/she will buy bread and the probability is about 33.75\%.

To sum up, this a simple recommendation algorithm. By this way, we can
obtain which products have a strong relationship, then bundling these
products as a whole to recommend to a customer, which will help the
bakery to increase the sale volume.

\hypertarget{visualisation}{%
\subsubsection{Visualisation}\label{visualisation}}

A table is shown below.

    \begin{tcolorbox}[breakable, size=fbox, boxrule=1pt, pad at break*=1mm,colback=cellbackground, colframe=cellborder]
\prompt{In}{incolor}{144}{\boxspacing}
\begin{Verbatim}[commandchars=\\\{\}]
\PY{n}{correction\PYZus{}df} \PY{o}{=} \PY{n}{apriori\PYZus{}model\PYZus{}algorithm}\PY{p}{(}\PY{n}{trans\PYZus{}items}\PY{p}{)}
\PY{n}{display}\PY{p}{(}\PY{n}{correction\PYZus{}df}\PY{p}{)}
\end{Verbatim}
\end{tcolorbox}

    
    \begin{verbatim}
         antecedents consequents  antecedent support  consequent support  \
15           (Toast)    (Coffee)            0.033272            0.478881   
13  (Spanish Brunch)    (Coffee)            0.017608            0.478881   
7        (Medialuna)    (Coffee)            0.062763            0.478881   
9           (Pastry)    (Coffee)            0.086745            0.478881   
6            (Juice)    (Coffee)            0.037809            0.478881   
0        (Alfajores)    (Coffee)            0.035757            0.478881   
10        (Sandwich)    (Coffee)            0.071081            0.478881   
3             (Cake)    (Coffee)            0.103813            0.478881   
11           (Scone)    (Coffee)            0.034784            0.478881   
4          (Cookies)    (Coffee)            0.054445            0.478881   
5    (Hot chocolate)    (Coffee)            0.059198            0.478881   
2          (Brownie)    (Coffee)            0.040402            0.478881   
8           (Muffin)    (Coffee)            0.038025            0.478881   
12            (Soup)    (Coffee)            0.034352            0.478881   
16     (Bread, Cake)    (Coffee)            0.023334            0.478881   
18       (Cake, Tea)    (Coffee)            0.023658            0.478881   
17   (Bread, Pastry)    (Coffee)            0.029275            0.478881   
14             (Tea)    (Coffee)            0.142811            0.478881   
1           (Pastry)     (Bread)            0.086745            0.328184   

     support  confidence      lift  leverage  conviction  
15  0.023226    0.698052  1.457674  0.007292    1.725857  
13  0.010479    0.595092  1.242672  0.002046    1.287006  
7   0.035541    0.566265  1.182476  0.005485    1.201469  
9   0.047964    0.552927  1.154622  0.006423    1.165623  
6   0.020417    0.540000  1.127629  0.002311    1.132868  
0   0.019229    0.537764  1.122961  0.002105    1.127388  
10  0.037917    0.533435  1.113919  0.003878    1.116926  
3   0.054769    0.527575  1.101684  0.005055    1.103074  
11  0.018040    0.518634  1.083012  0.001383    1.082583  
4   0.028087    0.515873  1.077247  0.002014    1.076410  
5   0.029923    0.505474  1.055533  0.001574    1.053776  
2   0.019877    0.491979  1.027351  0.000529    1.025782  
8   0.018689    0.491477  1.026304  0.000479    1.024771  
12  0.015880    0.462264  0.965301 -0.000571    0.969099  
16  0.010154    0.435185  0.908755 -0.001020    0.922637  
18  0.010046    0.424658  0.886771 -0.001283    0.905755  
17  0.011235    0.383764  0.801376 -0.002785    0.845648  
14  0.050232    0.351740  0.734504 -0.018157    0.803873  
1   0.029275    0.337484  1.028339  0.000807    1.014038  
    \end{verbatim}

    
    \hypertarget{conclusion-5-marks}{%
\section{Conclusion (5 marks)}\label{conclusion-5-marks}}

\hypertarget{acheivements}{%
\subsubsection{Acheivements}\label{acheivements}}

As we had expected, we count the total sales of every day and every
month from 2016-10-31 to 2017-04-11, and we can see the sales trend of
every day and every month. This will help us have a good track on the
business situation of the store. Besides, we can get which items are
sold in high volume based on the history data. Finally, by finding the
relationship between different items, we can recommend the product to a
customer.

\hypertarget{limitations}{%
\subsubsection{Limitations}\label{limitations}}

There are two limitations to this project.

\begin{itemize}
\tightlist
\item
  The project was limited to a small dataset, which may affect the
  accuracy of the recommendation.
\item
  There are only five columns in this data set. If we can know more
  information, such as the price of every item, the category of every
  item etc., will be better.
\end{itemize}

\hypertarget{future-work}{%
\subsubsection{Future Work}\label{future-work}}

In future work, we hope to collect more and more diverse data sets. And
then, we will improve this system through other common algorithms and
machine learning algorithms.

    \hypertarget{grading}{%
\section{Grading}\label{grading}}

\emph{Feedback and marks will be given here.}

\hypertarget{feedback}{%
\subsubsection{Feedback}\label{feedback}}

\hypertarget{marks}{%
\subsubsection{Marks}\label{marks}}

    \begin{tcolorbox}[breakable, size=fbox, boxrule=1pt, pad at break*=1mm,colback=cellbackground, colframe=cellborder]
\prompt{In}{incolor}{14}{\boxspacing}
\begin{Verbatim}[commandchars=\\\{\}]
\PY{n}{DATA}   \PY{o}{=} \PY{l+m+mi}{10}
\PY{n}{AIMS}   \PY{o}{=}  \PY{l+m+mi}{5}
\PY{n}{DESIGN} \PY{o}{=} \PY{l+m+mi}{5}

\PY{n}{CODE} \PY{o}{=} \PY{l+m+mi}{15}

\PY{n}{OUTCOME\PYZus{}EXPLANATION}   \PY{o}{=} \PY{l+m+mi}{10}
\PY{n}{OUTCOME\PYZus{}VISUALISATION} \PY{o}{=} \PY{l+m+mi}{10}

\PY{n}{CONCLUSION} \PY{o}{=} \PY{l+m+mi}{5}

\PY{n}{TOTAL} \PY{o}{=} \PY{p}{(} \PY{n}{DATA} \PY{o}{+} \PY{n}{AIMS} \PY{o}{+} \PY{n}{DESIGN} \PY{o}{+} \PY{n}{CODE} 
          \PY{o}{+} \PY{n}{OUTCOME\PYZus{}VISUALISATION} \PY{o}{+} \PY{n}{OUTCOME\PYZus{}VISUALISATION}
          \PY{o}{+} \PY{n}{CONCLUSION} \PY{p}{)}
\PY{n}{TOTAL}
\end{Verbatim}
\end{tcolorbox}

            \begin{tcolorbox}[breakable, size=fbox, boxrule=.5pt, pad at break*=1mm, opacityfill=0]
\prompt{Out}{outcolor}{14}{\boxspacing}
\begin{Verbatim}[commandchars=\\\{\}]
60
\end{Verbatim}
\end{tcolorbox}
        
    \begin{tcolorbox}[breakable, size=fbox, boxrule=1pt, pad at break*=1mm,colback=cellbackground, colframe=cellborder]
\prompt{In}{incolor}{ }{\boxspacing}
\begin{Verbatim}[commandchars=\\\{\}]

\end{Verbatim}
\end{tcolorbox}


    % Add a bibliography block to the postdoc
    
    
    
\end{document}
